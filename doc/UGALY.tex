% generated by GAPDoc2LaTeX from XML source (Frank Luebeck)
\documentclass[a4paper,11pt]{report}

\usepackage[top=37mm,bottom=37mm,left=27mm,right=27mm]{geometry}
\sloppy
\pagestyle{myheadings}
\usepackage{amssymb}
\usepackage[utf8]{inputenc}
\usepackage{makeidx}
\makeindex
\usepackage{color}
\definecolor{FireBrick}{rgb}{0.5812,0.0074,0.0083}
\definecolor{RoyalBlue}{rgb}{0.0236,0.0894,0.6179}
\definecolor{RoyalGreen}{rgb}{0.0236,0.6179,0.0894}
\definecolor{RoyalRed}{rgb}{0.6179,0.0236,0.0894}
\definecolor{LightBlue}{rgb}{0.8544,0.9511,1.0000}
\definecolor{Black}{rgb}{0.0,0.0,0.0}

\definecolor{linkColor}{rgb}{0.0,0.0,0.554}
\definecolor{citeColor}{rgb}{0.0,0.0,0.554}
\definecolor{fileColor}{rgb}{0.0,0.0,0.554}
\definecolor{urlColor}{rgb}{0.0,0.0,0.554}
\definecolor{promptColor}{rgb}{0.0,0.0,0.589}
\definecolor{brkpromptColor}{rgb}{0.589,0.0,0.0}
\definecolor{gapinputColor}{rgb}{0.589,0.0,0.0}
\definecolor{gapoutputColor}{rgb}{0.0,0.0,0.0}

%%  for a long time these were red and blue by default,
%%  now black, but keep variables to overwrite
\definecolor{FuncColor}{rgb}{0.0,0.0,0.0}
%% strange name because of pdflatex bug:
\definecolor{Chapter }{rgb}{0.0,0.0,0.0}
\definecolor{DarkOlive}{rgb}{0.1047,0.2412,0.0064}


\usepackage{fancyvrb}

\usepackage{mathptmx,helvet}
\usepackage[T1]{fontenc}
\usepackage{textcomp}


\usepackage[
            pdftex=true,
            bookmarks=true,        
            a4paper=true,
            pdftitle={Written with GAPDoc},
            pdfcreator={LaTeX with hyperref package / GAPDoc},
            colorlinks=true,
            backref=page,
            breaklinks=true,
            linkcolor=linkColor,
            citecolor=citeColor,
            filecolor=fileColor,
            urlcolor=urlColor,
            pdfpagemode={UseNone}, 
           ]{hyperref}

\newcommand{\maintitlesize}{\fontsize{50}{55}\selectfont}

% write page numbers to a .pnr log file for online help
\newwrite\pagenrlog
\immediate\openout\pagenrlog =\jobname.pnr
\immediate\write\pagenrlog{PAGENRS := [}
\newcommand{\logpage}[1]{\protect\write\pagenrlog{#1, \thepage,}}
%% were never documented, give conflicts with some additional packages

\newcommand{\GAP}{\textsf{GAP}}

%% nicer description environments, allows long labels
\usepackage{enumitem}
\setdescription{style=nextline}

%% depth of toc
\setcounter{tocdepth}{1}





%% command for ColorPrompt style examples
\newcommand{\gapprompt}[1]{\color{promptColor}{\bfseries #1}}
\newcommand{\gapbrkprompt}[1]{\color{brkpromptColor}{\bfseries #1}}
\newcommand{\gapinput}[1]{\color{gapinputColor}{#1}}


\begin{document}

\logpage{[ 0, 0, 0 ]}
\begin{titlepage}
\mbox{}\vfill

\begin{center}{\maintitlesize \textbf{ \textsf{UGALY} \mbox{}}}\\
\vfill

\hypersetup{pdftitle= \textsf{UGALY} }
\markright{\scriptsize \mbox{}\hfill  \textsf{UGALY}  \hfill\mbox{}}
{\Huge \textbf{ Universal Groups Acting LocallY \mbox{}}}\\
\vfill

{\Huge  v1.0 \mbox{}}\\[1cm]
{ 10 November 2020 \mbox{}}\\[1cm]
\mbox{}\\[2cm]
{\Large \textbf{ Khalil Hannouch\\
   \mbox{}}}\\
{\Large \textbf{ Stephan Tornier\\
   \mbox{}}}\\
\hypersetup{pdfauthor= Khalil Hannouch\\
   ;  Stephan Tornier\\
   }
\end{center}\vfill

\mbox{}\\
{\mbox{}\\
\small \noindent \textbf{ Khalil Hannouch\\
   }  Email: \href{mailto://khalil.hannouch@newcastle.edu.au} {\texttt{khalil.hannouch@newcastle.edu.au}}\\
  Homepage: \href{https://www.newcastle.edu.au/profile/khalil-hannouch} {\texttt{https://www.newcastle.edu.au/profile/khalil-hannouch}}}\\
{\mbox{}\\
\small \noindent \textbf{ Stephan Tornier\\
   }  Email: \href{mailto://stephan.tornier@newcastle.edu.au} {\texttt{stephan.tornier@newcastle.edu.au}}\\
  Homepage: \href{https://www.newcastle.edu.au/profile/stephan-tornier} {\texttt{https://www.newcastle.edu.au/profile/stephan-tornier}}}\\
\end{titlepage}

\newpage\setcounter{page}{2}
{\small 
\section*{Abstract}
\logpage{[ 0, 0, 1 ]}
 \textsf{UGALY} (\textsc{U}niversal \textsc{G}roups \textsc{A}cting \textsc{L}ocall\textsc{Y}) is a \textsc{GAP} package that provides methods to create, analyse and find local actions of
universal groups acting on locally finite regular trees, following
Burger-Mozes and Tornier. \mbox{}}\\[1cm]
{\small 
\section*{Copyright}
\logpage{[ 0, 0, 2 ]}
 \textsf{UGALY} is free software; you can redistribute it and/or modify it under the terms of
the \href{http://www.fsf.org/licenses/gpl.html} {GNU General Public License} as published by the Free Software Foundation; either version 3 of the License,
or (at your option) any later version. \mbox{}}\\[1cm]
{\small 
\section*{Acknowledgements}
\logpage{[ 0, 0, 3 ]}
 The second author owes thanks to Marc Burger and George Willis for their
support and acknowledges contributions from the SNSF Doc.Mobility fellowship
172120 and the ARC Discovery Project DP120100996 to the development of an
early version of this codebase. In its present form, the development of \textsf{UGALY} was made possible by the ARC Laureate Fellowship FL170100032 and the ARC DECRA
Fellowship DE210100180. \mbox{}}\\[1cm]
\newpage

\def\contentsname{Contents\logpage{[ 0, 0, 4 ]}}

\tableofcontents
\newpage

     
\chapter{\textcolor{Chapter }{Introduction}}\label{Chapter_Introduction}
\logpage{[ 1, 0, 0 ]}
\hyperdef{L}{X7DFB63A97E67C0A1}{}
{
  

 Let $\Omega$ be a set of cardinality $d\in\mathbb{N}_{\ge 3}$ and let $T_{d}=(V,E)$ be the $d$-regular tree. We follow Serre's graph theory notation \cite{Ser80}. Given a subgroup $H$ of the automorphism group $\mathrm{Aut}(T_{d})$ of $T_{d}$, and a vertex $x\in V$, the stabilizer $H_{x}$ of $x$ in $H$ induces a permutation group on the set $E(x):=\{e\in E\mid o(e)=x\}$ of edges issuing from $x$. We say that $H$ is locally "P" if for every $x\in V$ said permutation group satisfies the property "P", e.g. being transitive,
semiprimitive, quasiprimitive or $2$-transitive. In \cite{BM00a}, Burger-Mozes develop a remarkable structure theory of closed, non-discrete,
locally quasiprimitive subgroups of $\mathrm{Aut}(T_{d})$, which resembles the theory of semisimple Lie groups. They complement this
structure theory with a particularly accessible class of subgroups of $\mathrm{Aut}(T_{d})$ with prescribed local action: Given $F\le\mathrm{Sym}(\Omega)$ their universal group $\mathrm{U}(F)$ is closed in $\mathrm{Aut}(T_{d})$, vertex-transitive, compactly generated and locally permutation isomorphic to $F$. It is discrete if and only if $F$ is semiregular. When $F$ is transitive, $\mathrm{U}(F)$ is maximal up to conjugation among vertex-transitive subgroups of $\mathrm{Aut}(T_{d})$ that are locally permutation isomorphic to $F$, hence \emph{universal}. 

 This construction was generalized by the second author in \cite{Tor20}: In the spirit of $k$-closures of groups acting on trees developed in \cite{BEW15}, we generalize the universal group construction by prescribing the local
action on balls of a given radius $k\in\mathbb{N}$, the Burger-Mozes construction corresponding to the case $k=1$. Fix a tree $B_{d,k}$ which is isomorphic to a ball of radius $k$ in the labelled tree $T_{d}$ and let $l_{x}^{k}:B(x,k)\to B_{d,k}$ ($x\in V$) be the unique label-respecting isomorphism. Then 
\[\sigma_{k}:\mathrm{Aut}(T_{d})\times V\to\mathrm{Aut}(B_{d,k}),\ (g,x)\to
l_{gx}^{k}\circ g\circ (l_{x}^{k})^{-1}\]
 captures the \emph{$k$-local action} of $g$ at the vertex $x\in V$. 

 With this we can make the following defintition: Let $F\!\le\!\mathrm{Aut}(B_{d,k})$. Define 
\[\mathrm{U}_{k}(F):=\{g\in\mathrm{Aut}(T_{d})\mid \forall x\in V:\
\sigma_{k}(g,x)\in F\}.\]
 

 While $\mathrm{U}_{k}(F)$ is always closed, vertex-transitive and compactly generated, other properties
of $\mathrm{U}(F)$ do \emph{not} carry over. Foremost, the group $\mathrm{U}_{k}(F)$ need not be locally action isomorphic to $F$ and we say that $F\le\mathrm{Aut}(B_{d,k})$ satisfies condition (C) if it is. This can be viewed as an interchangeability
condition on neighbouring local actions, see Section \ref{Section_condition_C}. There is also a discreteness condition (D) on $F\le\mathrm{Aut}(B_{d,k})$ in terms of certain stabilizers in $F$ under which $\mathrm{U}_{k}(F)$ is discrete, see Section \ref{Section_condition_D}. 

 \textsf{UGALY} provides methods to create, analyse and find local actions $F\le\mathrm{Aut}(B_{d,k})$ that satisfy condition (C) and/or (D), including the constructions $\Gamma$, $\Delta$, $\Phi$, $\Sigma$, and $\Pi$ developed in \cite{Tor20}. This package was developed within the \href{ https://zerodimensional.group/"} {Zero-Dimensional Symmetry Research Group} in the \href{ https://www.newcastle.edu.au/school/mathematical-and-physical-sciences} {School of Mathematical and Physical Sciences} at \href{ https://www.newcastle.edu.au/} {The University of Newcastle} as part of a project course taken by the first author, supervised by the
second author. }

   
\chapter{\textcolor{Chapter }{Preliminaries}}\label{Chapter_Preliminaries}
\logpage{[ 2, 0, 0 ]}
\hyperdef{L}{X8749E1888244CC3D}{}
{
  We recall the following notation from the Introduction which is essential
throughout this manual, cf. \cite{Tor20}. Let $\Omega$ be a set of cardinality $d\in\mathbb{N}_{\ge 3}$ and let $T_{d}=(V,E)$ denote the $d$-regular tree, following the graph theory notation in \cite{Ser80}. A \emph{labelling} $l$ of $T_{d}$ is a map $l:E\to\Omega$ such that for every $x\in V$ the restriction $l_{x}:E(x)\to\Omega,\ e\mapsto l(e)$ is a bijection, and $l(e)=l(\overline{e})$ for all $e\in E$. For every $k\in\mathbb{N}$, fix a tree $B_{d,k}$ which is isomorphic to a ball of radius $k$ around a vertex in $T_{d}$ and carry over the labelling of $T_{d}$ to $B_{d,k}$ via the chosen isomorphism. We denote the center of $B_{d,k}$ by $b$. 

 For every $x\in V$ there is a unique, label-respecting isomorphism $l_{x}^{k}:B(x,k)\to B_{d,k}$. We define the \emph{$k$-local action} $\sigma_{k}(g,x)\in\mathrm{Aut}(B_{d,k})$ of an automorphism $g\!\in\!\mathrm{Aut}(T_{d})$ at a vertex $x\in V$ via the map 
\[\sigma_{k}:\mathrm{Aut}(T_{d})\times V\to\mathrm{Aut}(B_{d,k}),
\sigma_{k}(g,x)\mapsto \sigma_{k}(g,x):=l_{gx}^{k}\circ g\circ
(l_{x}^{k})^{-1}.\]
 
\section{\textcolor{Chapter }{Local actions}}\label{Chapter_Preliminaries_Section_Local_actions}
\logpage{[ 2, 1, 0 ]}
\hyperdef{L}{X86AC008984A3489F}{}
{
  In this package, a local action $F\le\Aut(B_{d,k})$ are handled as objects of the category \texttt{IsLocalAction} (\ref{IsLocalAction}) and have several attributes and properties introduced throughout this manual.
Most importantly, a local action always stores the degree $d$ and the radius $k$ of the ball $B_{d,k}$ that it acts on. 

\subsection{\textcolor{Chapter }{IsLocalAction (for IsPermGroup)}}
\logpage{[ 2, 1, 1 ]}\nobreak
\hyperdef{L}{X7FCF15167D3A44B7}{}
{\noindent\textcolor{FuncColor}{$\triangleright$\enspace\texttt{IsLocalAction({\mdseries\slshape arg})\index{IsLocalAction@\texttt{IsLocalAction}!for IsPermGroup}
\label{IsLocalAction:for IsPermGroup}
}\hfill{\scriptsize (filter)}}\\
\textbf{\indent Returns:\ }
\texttt{true} or \texttt{false} 



 Groups acting on the trees $B_{d,k}$ are stored together with their degree (\texttt{LocalActionDegree} (\ref{LocalActionDegree})), radius (\texttt{LocalActionRadius} (\ref{LocalActionRadius})) and other attributes in this category. 

 }

 

 
\begin{Verbatim}[commandchars=!@|,fontsize=\small,frame=single,label=Example]
  to do
\end{Verbatim}
 

\subsection{\textcolor{Chapter }{LocalAction (for IsInt, IsInt, IsPermGroup)}}
\logpage{[ 2, 1, 2 ]}\nobreak
\hyperdef{L}{X81135CA77A3C0F4E}{}
{\noindent\textcolor{FuncColor}{$\triangleright$\enspace\texttt{LocalAction({\mdseries\slshape d, k, F})\index{LocalAction@\texttt{LocalAction}!for IsInt, IsInt, IsPermGroup}
\label{LocalAction:for IsInt, IsInt, IsPermGroup}
}\hfill{\scriptsize (operation)}}\\
\textbf{\indent Returns:\ }
 the regular rooted tree group $G$ as an object of the category \texttt{IsLocalAction} (\ref{IsLocalAction}), checking that \mbox{\texttt{\mdseries\slshape F}} is indeed a subgroup of $\mathrm{Aut}(B_{d,k})$. 



 The arguments of this method are a degree \mbox{\texttt{\mdseries\slshape d}} $\in\mathbb{N}_{\ge 3}$, a radius \mbox{\texttt{\mdseries\slshape k}} $\in\mathbb{N}_{0}$ and a group \mbox{\texttt{\mdseries\slshape F}} $\le\mathrm{Aut}(B_{d,k})$. 

 }

 

 
\begin{Verbatim}[commandchars=!@|,fontsize=\small,frame=single,label=Example]
  to do
\end{Verbatim}
 

\subsection{\textcolor{Chapter }{LocalActionNC (for IsInt, IsInt, IsPermGroup)}}
\logpage{[ 2, 1, 3 ]}\nobreak
\hyperdef{L}{X84D5C421864EB7FD}{}
{\noindent\textcolor{FuncColor}{$\triangleright$\enspace\texttt{LocalActionNC({\mdseries\slshape d, k, F})\index{LocalActionNC@\texttt{LocalActionNC}!for IsInt, IsInt, IsPermGroup}
\label{LocalActionNC:for IsInt, IsInt, IsPermGroup}
}\hfill{\scriptsize (operation)}}\\
\textbf{\indent Returns:\ }
 the regular rooted tree group $G$ as an object of the category \texttt{IsLocalAction} (\ref{IsLocalAction}), without checking that \mbox{\texttt{\mdseries\slshape F}} is indeed a subgroup of $\mathrm{Aut}(B_{d,k})$. 



 The arguments of this method are a degree \mbox{\texttt{\mdseries\slshape d}} $\in\mathbb{N}_{\ge 3}$, a radius \mbox{\texttt{\mdseries\slshape k}} $\in\mathbb{N}_{0}$ and a group \mbox{\texttt{\mdseries\slshape F}} $\le\mathrm{Aut}(B_{d,k})$. 

 }

 

 
\begin{Verbatim}[commandchars=!@|,fontsize=\small,frame=single,label=Example]
  to do
\end{Verbatim}
 

\subsection{\textcolor{Chapter }{LocalActionDegree (for IsLocalAction)}}
\logpage{[ 2, 1, 4 ]}\nobreak
\hyperdef{L}{X8321CC72807D1096}{}
{\noindent\textcolor{FuncColor}{$\triangleright$\enspace\texttt{LocalActionDegree({\mdseries\slshape F})\index{LocalActionDegree@\texttt{LocalActionDegree}!for IsLocalAction}
\label{LocalActionDegree:for IsLocalAction}
}\hfill{\scriptsize (attribute)}}\\
\textbf{\indent Returns:\ }
 the degree \mbox{\texttt{\mdseries\slshape d}} of the ball $B_{d,k}$ that $F$ is acting on. 



 The argument of this attribute is a local action \mbox{\texttt{\mdseries\slshape F}} $\le\mathrm{Aut}(B_{d,k})$ (\texttt{IsLocalAction} (\ref{IsLocalAction})). 

 }

 

 
\begin{Verbatim}[commandchars=!@|,fontsize=\small,frame=single,label=Example]
  to do
\end{Verbatim}
 

\subsection{\textcolor{Chapter }{LocalActionRadius (for IsLocalAction)}}
\logpage{[ 2, 1, 5 ]}\nobreak
\hyperdef{L}{X7BF3EE4D794F8276}{}
{\noindent\textcolor{FuncColor}{$\triangleright$\enspace\texttt{LocalActionRadius({\mdseries\slshape F})\index{LocalActionRadius@\texttt{LocalActionRadius}!for IsLocalAction}
\label{LocalActionRadius:for IsLocalAction}
}\hfill{\scriptsize (attribute)}}\\
\textbf{\indent Returns:\ }
 the radius \mbox{\texttt{\mdseries\slshape k}} of the ball $B_{d,k}$ that $F$ is acting on. 



 The argument of this attribute is a local action \mbox{\texttt{\mdseries\slshape F}} $\le\mathrm{Aut}(B_{d,k})$ (\texttt{IsLocalAction} (\ref{IsLocalAction})). 

 }

 

 
\begin{Verbatim}[commandchars=!@|,fontsize=\small,frame=single,label=Example]
  to do
\end{Verbatim}
 

\subsection{\textcolor{Chapter }{LocalAction (for r, d, k, aut, addr)}}
\logpage{[ 2, 1, 6 ]}\nobreak
\hyperdef{L}{X7E0E11FC802B5210}{}
{\noindent\textcolor{FuncColor}{$\triangleright$\enspace\texttt{LocalAction({\mdseries\slshape r, d, k, aut, addr})\index{LocalAction@\texttt{LocalAction}!for r, d, k, aut, addr}
\label{LocalAction:for r, d, k, aut, addr}
}\hfill{\scriptsize (operation)}}\\
\textbf{\indent Returns:\ }
 the \mbox{\texttt{\mdseries\slshape r}}-local action $\sigma_{r}($\mbox{\texttt{\mdseries\slshape aut}},\mbox{\texttt{\mdseries\slshape addr}}$)$ of the automorphism \mbox{\texttt{\mdseries\slshape aut}} of $B_{d,k}$ at the vertex represented by the address \mbox{\texttt{\mdseries\slshape addr}}. 



 The arguments of this method are a radius \mbox{\texttt{\mdseries\slshape r}}, a degree \mbox{\texttt{\mdseries\slshape d}} $\in\mathbb{N}_{\ge 3}$, a radius \mbox{\texttt{\mdseries\slshape k}} $\in\mathbb{N}$, an automorphism \mbox{\texttt{\mdseries\slshape aut}} of $B_{d,k}$, and an address \mbox{\texttt{\mdseries\slshape addr}}. 

 }

 

 
\begin{Verbatim}[commandchars=!@|,fontsize=\small,frame=single,label=Example]
  !gapprompt@gap>| !gapinput@a:=(1,3,5)(2,4,6);; a in AutB(3,2);|
  true
  !gapprompt@gap>| !gapinput@LocalAction(2,3,2,a,[]);|
  (1,3,5)(2,4,6)
  !gapprompt@gap>| !gapinput@LocalAction(1,3,2,a,[]);|
  (1,2,3)
  !gapprompt@gap>| !gapinput@LocalAction(1,3,2,a,[1]);|
  (1,2)
\end{Verbatim}
 

 
\begin{Verbatim}[commandchars=!@|,fontsize=\small,frame=single,label=Example]
  !gapprompt@gap>| !gapinput@b:=Random(AutB(3,4));|
  (1,20,4,17,2,19,3,18)(5,22,8,23,6,21,7,24)(9,10)(13,16,14,15)
  !gapprompt@gap>| !gapinput@LocalAction(2,3,4,b,[3,1]);|
  (1,4)(2,3)
  !gapprompt@gap>| !gapinput@LocalAction(3,3,4,b,[3,1]);|
  Error, the sum of input argument r=3 and the length of input argument
  addr=[ 3, 1 ] must not exceed input argument k=4
\end{Verbatim}
 

\subsection{\textcolor{Chapter }{Projection (for F, r)}}
\logpage{[ 2, 1, 7 ]}\nobreak
\hyperdef{L}{X7BE35375787753EE}{}
{\noindent\textcolor{FuncColor}{$\triangleright$\enspace\texttt{Projection({\mdseries\slshape F, r})\index{Projection@\texttt{Projection}!for F, r}
\label{Projection:for F, r}
}\hfill{\scriptsize (operation)}}\\
\textbf{\indent Returns:\ }
 the restriction of the projection map $\mathrm{Aut}(B_{d,k})\to\mathrm{Aut}(B_{d,r})$ to \mbox{\texttt{\mdseries\slshape F}}. 



 The arguments of this method are a local action \mbox{\texttt{\mdseries\slshape F}} $\le\mathrm{Aut}(B_{d,k})$, and a projection radius \mbox{\texttt{\mdseries\slshape r}} $\le$ \mbox{\texttt{\mdseries\slshape k}}. 

 }

 

 
\begin{Verbatim}[commandchars=!@|,fontsize=\small,frame=single,label=Example]
  !gapprompt@gap>| !gapinput@F:=GAMMA(4,3,SymmetricGroup(3));|
  Group([ (1,16,19)(2,15,20)(3,13,18)(4,14,17)(5,10,23)(6,9,24)(7,12,22)
    (8,11,21), (1,9)(2,10)(3,12)(4,11)(5,15)(6,16)(7,13)(8,14)(17,21)(18,22)
    (19,24)(20,23) ])
  !gapprompt@gap>| !gapinput@pr:=Projection(3,4,F,2);|
  <action homomorphism>
  !gapprompt@gap>| !gapinput@a:=Random(F);; Image(pr,a);|
  (1,4,5)(2,3,6)
\end{Verbatim}
 

\subsection{\textcolor{Chapter }{ImageOfProjection}}
\logpage{[ 2, 1, 8 ]}\nobreak
\hyperdef{L}{X87A13DDE8321BEF3}{}
{\noindent\textcolor{FuncColor}{$\triangleright$\enspace\texttt{ImageOfProjection({\mdseries\slshape F, r})\index{ImageOfProjection@\texttt{ImageOfProjection}}
\label{ImageOfProjection}
}\hfill{\scriptsize (function)}}\\
\textbf{\indent Returns:\ }
 the local action $\sigma_{r}(F,b)\le\mathrm{Aut}(B_{d,r})$. 



 The arguments of this method are a local action \mbox{\texttt{\mdseries\slshape F}} $\le\mathrm{Aut}(B_{d,k})$, and a projection radius \mbox{\texttt{\mdseries\slshape r}} $\le$ \mbox{\texttt{\mdseries\slshape k}}. This method uses \texttt{LocalAction} (\ref{LocalAction:for r, d, k, aut, addr}) on generators rather than \texttt{Projection} (\ref{Projection:for F, r}) on the group to compute the image. 

 }

 

 
\begin{Verbatim}[commandchars=!@|,fontsize=\small,frame=single,label=Example]
  !gapprompt@gap>| !gapinput@AutB(3,2);|
  Group([ (1,2), (3,4), (5,6), (1,3,5)(2,4,6), (1,3)(2,4) ])
  !gapprompt@gap>| !gapinput@ImageOfProjection(3,2,AutB(3,2),1);|
  Group([ (), (), (), (1,2,3), (1,2) ])
\end{Verbatim}
 }

 
\section{\textcolor{Chapter }{Finite balls}}\label{Chapter_Preliminaries_Section_Finite_balls}
\logpage{[ 2, 2, 0 ]}
\hyperdef{L}{X855A2B187C52B82A}{}
{
  The automorphism groups of the finite labelled balls $B_{d,k}$ lie at the center of this package. The method \texttt{AutB} (\ref{AutB}) produces these automorphism groups as iterated wreath products. The result is
a permutation group on the set of leaves of $B_{d,k}$. 

\subsection{\textcolor{Chapter }{AutB}}
\logpage{[ 2, 2, 1 ]}\nobreak
\hyperdef{L}{X7B766DE9824037BB}{}
{\noindent\textcolor{FuncColor}{$\triangleright$\enspace\texttt{AutB({\mdseries\slshape d, k})\index{AutB@\texttt{AutB}}
\label{AutB}
}\hfill{\scriptsize (function)}}\\
\textbf{\indent Returns:\ }
 the local action $\mathrm{Aut}(B_{d,k})$ as a permutation group of the $d\cdot (d-1)^{k-1}$ leaves of $B_{d,k}$. 



 The arguments of this method are a degree \mbox{\texttt{\mdseries\slshape d}} $\in\mathbb{N}_{\ge 3}$ and a radius \mbox{\texttt{\mdseries\slshape k}} $\in\mathbb{N}_{0}$. 

 }

 

 
\begin{Verbatim}[commandchars=!@|,fontsize=\small,frame=single,label=Example]
  !gapprompt@gap>| !gapinput@G:=AutB(3,2);|
  Group([ (1,2), (3,4), (5,6), (1,3,5)(2,4,6), (1,3)(2,4) ])
  !gapprompt@gap>| !gapinput@Size(G);|
  48
\end{Verbatim}
 }

 
\section{\textcolor{Chapter }{Addresses and leaves}}\label{Chapter_Preliminaries_Section_Addresses_and_leaves}
\logpage{[ 2, 3, 0 ]}
\hyperdef{L}{X7A77C1B579101B7C}{}
{
  The vertices at distance $n$ from the center $b$ of $B_{d,k}$ are addressed as elements of the set 
\[\Omega^{(n)}:=\{(\omega_{1},\ldots,\omega_{n})\in\Omega^{n}\mid \forall
l\in\{1,\ldots,n-1\}:\ \omega_{l}\neq\omega_{l+1}\},\]
 i.e. as lists of length $n$ of elements from \texttt{[1..d]} such that no two consecutive entries are equal. They are ordered according to
the lexicographic order on $\Omega^{(n)}$. The center $b$ itself is addressed by the empty list \texttt{[]}. Note that the leaves of $B_{d,k}$ correspond to elements of $\Omega^{(k)}$. 

\subsection{\textcolor{Chapter }{Addresses}}
\logpage{[ 2, 3, 1 ]}\nobreak
\hyperdef{L}{X843C3D8E82F0D51F}{}
{\noindent\textcolor{FuncColor}{$\triangleright$\enspace\texttt{Addresses({\mdseries\slshape d, k})\index{Addresses@\texttt{Addresses}}
\label{Addresses}
}\hfill{\scriptsize (function)}}\\
\textbf{\indent Returns:\ }
 a list of all addresses of vertices in $B_{d,k}$ in ascending order with respect to length, lexicographically ordered within
each level. See \texttt{AddressOfLeaf} (\ref{AddressOfLeaf}) and \texttt{LeafOfAddress} (\ref{LeafOfAddress}) for the correspondence between the leaves of $B_{d,k}$ and addresses of length \mbox{\texttt{\mdseries\slshape k}}. 



 The arguments of this method are a degree \mbox{\texttt{\mdseries\slshape d}} $\in\mathbb{N}_{\ge 3}$ and a radius \mbox{\texttt{\mdseries\slshape k}} $\in\mathbb{N}_{0}$. 

 }

 

 
\begin{Verbatim}[commandchars=!@|,fontsize=\small,frame=single,label=Example]
  !gapprompt@gap>| !gapinput@Addresses(3,1);|
  [ [  ], [ 1 ], [ 2 ], [ 3 ] ]
  !gapprompt@gap>| !gapinput@Addresses(3,2);|
  [ [  ], [ 1 ], [ 2 ], [ 3 ], [ 1, 2 ], [ 1, 3 ], [ 2, 1 ], [ 2, 3 ], 
  [ 3, 1 ], [ 3, 2 ] ]
\end{Verbatim}
 

\subsection{\textcolor{Chapter }{LeafAddresses}}
\logpage{[ 2, 3, 2 ]}\nobreak
\hyperdef{L}{X868811D87FEB1AA6}{}
{\noindent\textcolor{FuncColor}{$\triangleright$\enspace\texttt{LeafAddresses({\mdseries\slshape d, k})\index{LeafAddresses@\texttt{LeafAddresses}}
\label{LeafAddresses}
}\hfill{\scriptsize (function)}}\\
\textbf{\indent Returns:\ }
 a list of addresses of the leaves of $B_{d,k}$ in lexicographic order. 



 The arguments of this method are a degree \mbox{\texttt{\mdseries\slshape d}} $\in\mathbb{N}_{\ge 3}$ and a radius \mbox{\texttt{\mdseries\slshape k}} $\in\mathbb{N}_{0}$. 

 }

 

 
\begin{Verbatim}[commandchars=!@|,fontsize=\small,frame=single,label=Example]
  !gapprompt@gap>| !gapinput@LeafAddresses(3,2);|
  [ [ 1, 2 ], [ 1, 3 ], [ 2, 1 ], [ 2, 3 ], [ 3, 1 ], [ 3, 2 ] ]
\end{Verbatim}
 

\subsection{\textcolor{Chapter }{AddressOfLeaf}}
\logpage{[ 2, 3, 3 ]}\nobreak
\hyperdef{L}{X78379A547ED7A317}{}
{\noindent\textcolor{FuncColor}{$\triangleright$\enspace\texttt{AddressOfLeaf({\mdseries\slshape d, k, lf})\index{AddressOfLeaf@\texttt{AddressOfLeaf}}
\label{AddressOfLeaf}
}\hfill{\scriptsize (function)}}\\
\textbf{\indent Returns:\ }
 the address of the leaf \mbox{\texttt{\mdseries\slshape lf}} of $B_{d,k}$ with respect to the lexicographic order. 



 The arguments of this method are a degree \mbox{\texttt{\mdseries\slshape d}} $\in\mathbb{N}_{\ge 3}$, a radius \mbox{\texttt{\mdseries\slshape k}} $\in\mathbb{N}$, and a leaf \mbox{\texttt{\mdseries\slshape lf}} of $B_{d,k}$. 

 }

 

 
\begin{Verbatim}[commandchars=!@|,fontsize=\small,frame=single,label=Example]
  !gapprompt@gap>| !gapinput@AddressOfLeaf(3,2,1);|
  [ 1, 2 ]
  !gapprompt@gap>| !gapinput@AddressOfLeaf(3,3,1);|
  [ 1, 2, 1 ]
\end{Verbatim}
 

\subsection{\textcolor{Chapter }{LeafOfAddress}}
\logpage{[ 2, 3, 4 ]}\nobreak
\hyperdef{L}{X7E43E2B87B97A9BE}{}
{\noindent\textcolor{FuncColor}{$\triangleright$\enspace\texttt{LeafOfAddress({\mdseries\slshape d, k, addr})\index{LeafOfAddress@\texttt{LeafOfAddress}}
\label{LeafOfAddress}
}\hfill{\scriptsize (function)}}\\
\textbf{\indent Returns:\ }
 the smallest leaf (integer) whose address has \mbox{\texttt{\mdseries\slshape addr}} as a prefix. 



 The arguments of this method are a degree \mbox{\texttt{\mdseries\slshape d}} $\in\mathbb{N}_{\ge 3}$, a radius \mbox{\texttt{\mdseries\slshape k}} $\in\mathbb{N}$, and an address \mbox{\texttt{\mdseries\slshape addr}}. 

 }

 

 
\begin{Verbatim}[commandchars=!@|,fontsize=\small,frame=single,label=Example]
  !gapprompt@gap>| !gapinput@LeafOfAddress(3,2,[1,2]);|
  1
  !gapprompt@gap>| !gapinput@LeafOfAddress(3,2,[3]);|
  5
  !gapprompt@gap>| !gapinput@LeafOfAddress(3,2,[]);|
  1
\end{Verbatim}
 

\subsection{\textcolor{Chapter }{ImageAddress}}
\logpage{[ 2, 3, 5 ]}\nobreak
\hyperdef{L}{X871D6B8783E17AB8}{}
{\noindent\textcolor{FuncColor}{$\triangleright$\enspace\texttt{ImageAddress({\mdseries\slshape d, k, aut, addr})\index{ImageAddress@\texttt{ImageAddress}}
\label{ImageAddress}
}\hfill{\scriptsize (function)}}\\
\textbf{\indent Returns:\ }
 the address of the image of the vertex represented by the address \mbox{\texttt{\mdseries\slshape addr}} under the automorphism \mbox{\texttt{\mdseries\slshape aut}} of $B_{d,k}$. 



 The arguments of this method are a degree \mbox{\texttt{\mdseries\slshape d}} $\in\mathbb{N}_{\ge 3}$, a radius \mbox{\texttt{\mdseries\slshape k}} $\in\mathbb{N}$, an automorphism \mbox{\texttt{\mdseries\slshape aut}} of $B_{d,k}$, and an address \mbox{\texttt{\mdseries\slshape addr}}. 

 }

 

 
\begin{Verbatim}[commandchars=!@|,fontsize=\small,frame=single,label=Example]
  !gapprompt@gap>| !gapinput@ImageAddress(3,2,(1,2),[1,2]);|
  [ 1, 3 ]
  !gapprompt@gap>| !gapinput@ImageAddress(3,2,(1,2),[1]);|
  [ 1 ]
\end{Verbatim}
 

\subsection{\textcolor{Chapter }{ComposeAddresses}}
\logpage{[ 2, 3, 6 ]}\nobreak
\hyperdef{L}{X862D09157F1D9D98}{}
{\noindent\textcolor{FuncColor}{$\triangleright$\enspace\texttt{ComposeAddresses({\mdseries\slshape addr1, addr2})\index{ComposeAddresses@\texttt{ComposeAddresses}}
\label{ComposeAddresses}
}\hfill{\scriptsize (function)}}\\
\textbf{\indent Returns:\ }
 the concatenation of the addresses \mbox{\texttt{\mdseries\slshape addr1}} and \mbox{\texttt{\mdseries\slshape addr2}} with reduction as per \cite[Section 3.2]{Tor20}. 



 The arguments of this method are two addresses \mbox{\texttt{\mdseries\slshape addr1}} and \mbox{\texttt{\mdseries\slshape addr2}}. 

 }

 

 
\begin{Verbatim}[commandchars=!@|,fontsize=\small,frame=single,label=Example]
  !gapprompt@gap>| !gapinput@ComposeAddresses([1,3],[2,1]);  |
  [ 1, 3, 2, 1 ]
  !gapprompt@gap>| !gapinput@ComposeAddresses([1,3,2],[2,1]);|
  [ 1, 3, 1 ]
\end{Verbatim}
 }

 }

   
\chapter{\textcolor{Chapter }{Compatibility}}\label{Chapter_Compatibility}
\logpage{[ 3, 0, 0 ]}
\hyperdef{L}{X7F4E157A827198EA}{}
{
  

 
\section{\textcolor{Chapter }{The compatibility condition (C)}}\label{Section_condition_C}
\logpage{[ 3, 1, 0 ]}
\hyperdef{L}{X81B0CAE97D161B97}{}
{
  A subgroup $F\le\mathrm{Aut}(B_{d,k})$ satifies the compatibility condition (C) if and only if if $\mathrm{U}_{k}(F)$ is locally action isomorphic to $F$, see \cite[Proposition 3.8]{Tor20}. The term \emph{compatibility} comes from the following translation of this condition into properties of the $(k-1)$-local actions of elements of $F$: The group $F$ satisfies (C) if and only if 
\[\forall \alpha\in F\ \forall\omega\in\Omega\ \exists\beta\in F:\
\sigma_{k-1}(\alpha,b)=\sigma_{k-1}(\beta,b_{\omega}),\
\sigma_{k-1}(\alpha,b_{\omega})=\sigma_{k-1}(\beta,b).\]
 }

 
\section{\textcolor{Chapter }{Compatible elements}}\label{Section_compatible_elements}
\logpage{[ 3, 2, 0 ]}
\hyperdef{L}{X82C839D7794DDBCD}{}
{
  This section is concerned with testing compatibility of two given elements (\texttt{AreCompatibleElements} (\ref{AreCompatibleElements})) and finding an/all elements that is/are compatible with a given one (\texttt{CompatibleElement} (\ref{CompatibleElement}), \texttt{CompatibilitySet} (\ref{CompatibilitySet})). 

\subsection{\textcolor{Chapter }{AreCompatibleElements}}
\logpage{[ 3, 2, 1 ]}\nobreak
\hyperdef{L}{X7A3E4DE2801A21B7}{}
{\noindent\textcolor{FuncColor}{$\triangleright$\enspace\texttt{AreCompatibleElements({\mdseries\slshape d, k, aut1, aut2, dir})\index{AreCompatibleElements@\texttt{AreCompatibleElements}}
\label{AreCompatibleElements}
}\hfill{\scriptsize (function)}}\\
\textbf{\indent Returns:\ }
 \texttt{true} if \mbox{\texttt{\mdseries\slshape aut1}} and \mbox{\texttt{\mdseries\slshape aut2}} are compatible with each other in direction \mbox{\texttt{\mdseries\slshape dir}}, and \texttt{false} otherwise. 



 The arguments of this method are a degree \mbox{\texttt{\mdseries\slshape d}} $\in\mathbb{N}_{\ge 3}$, a radius \mbox{\texttt{\mdseries\slshape k}} $\in\mathbb{N}$, two automorphisms \mbox{\texttt{\mdseries\slshape aut1}}, \mbox{\texttt{\mdseries\slshape aut2}} $\in\mathrm{Aut}(B_{d,k})$, and a direction \mbox{\texttt{\mdseries\slshape dir}} $\in$\texttt{[1..d]}. 

 }

 

 
\begin{Verbatim}[commandchars=!@|,fontsize=\small,frame=single,label=Example]
  !gapprompt@gap>| !gapinput@AreCompatibleElements(3,1,(1,2),(1,2,3),1);|
  true
  !gapprompt@gap>| !gapinput@AreCompatibleElements(3,1,(1,2),(1,2,3),2);|
  false
\end{Verbatim}
 

 
\begin{Verbatim}[commandchars=!@|,fontsize=\small,frame=single,label=Example]
  !gapprompt@gap>| !gapinput@a:=(1,3,5)(2,4,6);; a in AutB(3,2);|
  true
  !gapprompt@gap>| !gapinput@LocalAction(1,3,2,a,[]); LocalAction(1,3,2,a,[1]);|
  (1,2,3)
  (1,2)
  !gapprompt@gap>| !gapinput@b:=(1,4)(2,3);; b in AutB(3,2);|
  true
  !gapprompt@gap>| !gapinput@LocalAction(1,3,2,b,[]); LocalAction(1,3,2,b,[1]);|
  (1,2)
  (1,2,3)
  
  !gapprompt@gap>| !gapinput@AreCompatibleElements(3,2,a,b,1);|
  true
  !gapprompt@gap>| !gapinput@AreCompatibleElements(3,2,a,b,3);|
  false
\end{Verbatim}
 

\subsection{\textcolor{Chapter }{CompatibleElement}}
\logpage{[ 3, 2, 2 ]}\nobreak
\hyperdef{L}{X7EF32E607ABFB451}{}
{\noindent\textcolor{FuncColor}{$\triangleright$\enspace\texttt{CompatibleElement({\mdseries\slshape F, aut, dir})\index{CompatibleElement@\texttt{CompatibleElement}}
\label{CompatibleElement}
}\hfill{\scriptsize (function)}}\\
\textbf{\indent Returns:\ }
 an element of \mbox{\texttt{\mdseries\slshape F}} that is compatible with \mbox{\texttt{\mdseries\slshape aut}} in direction \mbox{\texttt{\mdseries\slshape dir}} if one exists, and \texttt{fail} otherwise. 



 The arguments of this method are a local action \mbox{\texttt{\mdseries\slshape F}} $\le\mathrm{Aut}(B_{d,k})$, an element \mbox{\texttt{\mdseries\slshape aut}} $\in$ \mbox{\texttt{\mdseries\slshape F}}, and a direction \mbox{\texttt{\mdseries\slshape dir}} $\in$\texttt{[1..d]}. 

 }

 

 
\begin{Verbatim}[commandchars=!@|,fontsize=\small,frame=single,label=Example]
  !gapprompt@gap>| !gapinput@a:=Random(AutB(5,1)); dir:=Random([1..5]);|
  (1,3,2,5)
  4
  !gapprompt@gap>| !gapinput@CompatibleElement(5,1,AutB(5,1),a,dir);|
  (1,3,2,5)
\end{Verbatim}
 

 
\begin{Verbatim}[commandchars=!@|,fontsize=\small,frame=single,label=Example]
  !gapprompt@gap>| !gapinput@a:=(1,3,5)(2,4,6);; a in AutB(3,2);|
  true
  !gapprompt@gap>| !gapinput@CompatibleElement(3,2,AutB(3,2),a,1);|
  (1,4,2,3)
\end{Verbatim}
 
\subsection{\textcolor{Chapter }{CompatibilitySet}}\label{CompatibilitySet}
\logpage{[ 3, 2, 3 ]}
\hyperdef{L}{X7ECC9E9982597E12}{}
{
\noindent\textcolor{FuncColor}{$\triangleright$\enspace\texttt{CompatibilitySet({\mdseries\slshape F, aut, dir})\index{CompatibilitySet@\texttt{CompatibilitySet}!for F, aut, dir}
\label{CompatibilitySet:for F, aut, dir}
}\hfill{\scriptsize (operation)}}\\
\noindent\textcolor{FuncColor}{$\triangleright$\enspace\texttt{CompatibilitySet({\mdseries\slshape F, aut, dirs})\index{CompatibilitySet@\texttt{CompatibilitySet}!for F, aut, dirs}
\label{CompatibilitySet:for F, aut, dirs}
}\hfill{\scriptsize (operation)}}\\


 

 
\begin{description}
\item[{for the arguments \mbox{\texttt{\mdseries\slshape F}}, \mbox{\texttt{\mdseries\slshape aut}}, \mbox{\texttt{\mdseries\slshape dir}}}]  Returns: the list of elements of \mbox{\texttt{\mdseries\slshape F}} that are compatible with \mbox{\texttt{\mdseries\slshape aut}} in direction \mbox{\texttt{\mdseries\slshape dir}}. 

 The arguments of this method are a local action \mbox{\texttt{\mdseries\slshape F}} of $\le\mathrm{Aut}(B_{d,k})$, an automorphism \mbox{\texttt{\mdseries\slshape aut}} $\in F$, and a direction \mbox{\texttt{\mdseries\slshape dir}} $\in$\texttt{[1..d]}. 
\item[{for the arguments \mbox{\texttt{\mdseries\slshape F}}, \mbox{\texttt{\mdseries\slshape aut}}, \mbox{\texttt{\mdseries\slshape dirs}}}]  Returns: the list of elements of \mbox{\texttt{\mdseries\slshape F}} that are compatible with \mbox{\texttt{\mdseries\slshape aut}} in all directions of \mbox{\texttt{\mdseries\slshape dirs}}. 

 The arguments of this method are a local action \mbox{\texttt{\mdseries\slshape F}} of $\le\mathrm{Aut}(B_{d,k})$, an automorphism \mbox{\texttt{\mdseries\slshape aut}} $\in F$, and a sublist of directions \mbox{\texttt{\mdseries\slshape dirs}} $\subseteq$\texttt{[1..d]}. 
\end{description}
 

 }

 

 
\begin{Verbatim}[commandchars=!@|,fontsize=\small,frame=single,label=Example]
  !gapprompt@gap>| !gapinput@F:=TransitiveGroup(4,3);|
  D(4)
  !gapprompt@gap>| !gapinput@aut:=(1,3);; aut in F;|
  true
  !gapprompt@gap>| !gapinput@CompatibilitySet(4,1,SymmetricGroup(4),aut,1);|
  RightCoset(Sym( [ 2 .. 4 ] ),(1,3))
  !gapprompt@gap>| !gapinput@CompatibilitySet(4,1,F,aut,1);|
  RightCoset(Group([ (2,4) ]),(1,3))
  !gapprompt@gap>| !gapinput@CompatibilitySet(4,1,F,aut,[1,3]);|
  RightCoset(Group([ (2,4) ]),(1,3))
  !gapprompt@gap>| !gapinput@CompatibilitySet(4,1,F,aut,[1,2]);|
  RightCoset(Group(()),(1,3))
\end{Verbatim}
 

 

\subsection{\textcolor{Chapter }{AssembleAutomorphism}}
\logpage{[ 3, 2, 4 ]}\nobreak
\hyperdef{L}{X837E15117DB5B1DE}{}
{\noindent\textcolor{FuncColor}{$\triangleright$\enspace\texttt{AssembleAutomorphism({\mdseries\slshape d, k, auts})\index{AssembleAutomorphism@\texttt{AssembleAutomorphism}}
\label{AssembleAutomorphism}
}\hfill{\scriptsize (function)}}\\
\textbf{\indent Returns:\ }
 the automorphism $($\texttt{aut}$,($\mbox{\texttt{\mdseries\slshape auts}}$[$\texttt{i}$])_{i=1}^{d})$ of $B_{d,k+1}$, where \texttt{aut} is implicit in $($\mbox{\texttt{\mdseries\slshape auts}}$[$\texttt{i}$])_{i=1}^{d}$. 



 The arguments of this method are a degree \mbox{\texttt{\mdseries\slshape d}} $\in\mathbb{N}_{\ge 3}$, a radius \mbox{\texttt{\mdseries\slshape k}} $\in\mathbb{N}$, and a list \mbox{\texttt{\mdseries\slshape auts}} of \mbox{\texttt{\mdseries\slshape d}} automorphisms $($\mbox{\texttt{\mdseries\slshape auts}}$[$\texttt{i}$])_{i=1}^{d}$ of $B_{d,k}$ which comes from an element $($\texttt{aut}$,($\mbox{\texttt{\mdseries\slshape auts}}$[$\texttt{i}$])_{i=1}^{d})$ of $\mathrm{Aut}(B_{d,k+1})$. 

 }

 

 
\begin{Verbatim}[commandchars=!@|,fontsize=\small,frame=single,label=Example]
  !gapprompt@gap>| !gapinput@aut:=Random(AutB(3,2));|
  (1,2)(3,6)(4,5)
  !gapprompt@gap>| !gapinput@auts:=[];;|
  !gapprompt@gap>| !gapinput@for i in [1..3] do auts[i]:=CompatibleElement(3,2,AutB(3,2),aut,i); od;|
  !gapprompt@gap>| !gapinput@auts;|
  [ (1,2)(3,5)(4,6), (1,3,5)(2,4,6), (1,5,3)(2,6,4) ]
  !gapprompt@gap>| !gapinput@a:=AssembleAutomorphism(3,2,auts);|
  (1,3)(2,4)(5,11)(6,12)(7,9)(8,10)
  !gapprompt@gap>| !gapinput@a in AutB(3,3); |
  true
  !gapprompt@gap>| !gapinput@LocalAction(2,3,3,a,[]);|
  (1,2)(3,6)(4,5)
\end{Verbatim}
 }

 
\section{\textcolor{Chapter }{Compatible subgroups}}\label{Chapter_Compatibility_Section_Compatible_subgroups}
\logpage{[ 3, 3, 0 ]}
\hyperdef{L}{X7967612B81C5074E}{}
{
  Using the methods of Section \ref{Section_compatible_elements}, this section provides methods to test groups for the compatibility condition
and search for compatible subgroups inside a given group, e.g. $\mathrm{Aut}(B_{d,k})$, or with a certain image under some projection. 

\subsection{\textcolor{Chapter }{MaximalCompatibleSubgroup (for IsLocalAction)}}
\logpage{[ 3, 3, 1 ]}\nobreak
\hyperdef{L}{X79D1BC54814345D4}{}
{\noindent\textcolor{FuncColor}{$\triangleright$\enspace\texttt{MaximalCompatibleSubgroup({\mdseries\slshape F})\index{MaximalCompatibleSubgroup@\texttt{MaximalCompatibleSubgroup}!for IsLocalAction}
\label{MaximalCompatibleSubgroup:for IsLocalAction}
}\hfill{\scriptsize (attribute)}}\\
\textbf{\indent Returns:\ }
The local action $C($\mbox{\texttt{\mdseries\slshape F}}$)\le\mathrm{Aut}(B_{d,k})$, which is the maximal compatible subgroup of \mbox{\texttt{\mdseries\slshape F}}. 



 The argument of this attribute is a local action \mbox{\texttt{\mdseries\slshape F}} $\le\mathrm{Aut}(B_{d,k})$ (\texttt{IsLocalAction} (\ref{IsLocalAction})). 

 }

 

 
\begin{Verbatim}[commandchars=!@|,fontsize=\small,frame=single,label=Example]
  !gapprompt@gap>| !gapinput@MaximalCompatibleSubgroup(3,1,Group((1,2)));|
  Group([ (1,2) ])
  !gapprompt@gap>| !gapinput@MaximalCompatibleSubgroup(3,2,Group((1,2)));|
  Group(())
\end{Verbatim}
 

\subsection{\textcolor{Chapter }{SatisfiesC (for IsLocalAction)}}
\logpage{[ 3, 3, 2 ]}\nobreak
\hyperdef{L}{X8302F28A85F1C4FE}{}
{\noindent\textcolor{FuncColor}{$\triangleright$\enspace\texttt{SatisfiesC({\mdseries\slshape F})\index{SatisfiesC@\texttt{SatisfiesC}!for IsLocalAction}
\label{SatisfiesC:for IsLocalAction}
}\hfill{\scriptsize (property)}}\\
\textbf{\indent Returns:\ }
\texttt{true} if \mbox{\texttt{\mdseries\slshape F}} satisfies the compatibility condition (C), and \texttt{false} otherwise. 



 The argument of this property is a local action \mbox{\texttt{\mdseries\slshape F}} $\le\mathrm{Aut}(B_{d,k})$ (\texttt{IsLocalAction} (\ref{IsLocalAction})). 

 }

 

 
\begin{Verbatim}[commandchars=!@|,fontsize=\small,frame=single,label=Example]
  !gapprompt@gap>| !gapinput@D:=DELTA(3,SymmetricGroup(3));|
  Group([ (1,3,6)(2,4,5), (1,3)(2,4), (1,2)(3,4)(5,6) ])
  !gapprompt@gap>| !gapinput@IsCompatible(3,2,D);|
  true
\end{Verbatim}
 

\subsection{\textcolor{Chapter }{CompatibleSubgroups}}
\logpage{[ 3, 3, 3 ]}\nobreak
\hyperdef{L}{X7FF0262182F3B9E8}{}
{\noindent\textcolor{FuncColor}{$\triangleright$\enspace\texttt{CompatibleSubgroups({\mdseries\slshape F})\index{CompatibleSubgroups@\texttt{CompatibleSubgroups}}
\label{CompatibleSubgroups}
}\hfill{\scriptsize (function)}}\\
\textbf{\indent Returns:\ }
the list of all compatible subgroups of \mbox{\texttt{\mdseries\slshape F}}. 



 The argument of this method is a local action \mbox{\texttt{\mdseries\slshape F}} $\le\mathrm{Aut}(B_{d,k})$. This method calls \texttt{AllSubgroups} on $F$ and is therefore slow. Use for instructional purposes on small examples only,
and use \texttt{ConjugacyClassRepsCompatibleSubgroups} (\ref{ConjugacyClassRepsCompatibleSubgroups}) or \texttt{ConjugacyClassRepsCompatibleSubgroupsWithProjection} (\ref{ConjugacyClassRepsCompatibleSubgroupsWithProjection}) for computations. 

 }

 

 
\begin{Verbatim}[commandchars=!@|,fontsize=\small,frame=single,label=Example]
  !gapprompt@gap>| !gapinput@G:=GAMMA(3,SymmetricGroup(3));|
  Group([ (1,4,5)(2,3,6), (1,3)(2,4)(5,6) ])
  !gapprompt@gap>| !gapinput@list:=CompatibleSubgroups(3,2,G);|
  [ Group(()), Group([ (1,2)(3,5)(4,6) ]), Group([ (1,3)(2,4)(5,6) ]), 
    Group([ (1,6)(2,5)(3,4) ]), Group([ (1,4,5)(2,3,6) ]), Group([ (1,4,5)
    (2,3,6), (1,3)(2,4)(5,6) ]) ]
  !gapprompt@gap>| !gapinput@Size(list);|
  6
  !gapprompt@gap>| !gapinput@Size(AllSubgroups(SymmetricGroup(3)));|
  6
\end{Verbatim}
 

\subsection{\textcolor{Chapter }{ConjugacyClassRepsCompatibleSubgroups (for IsLocalAction)}}
\logpage{[ 3, 3, 4 ]}\nobreak
\hyperdef{L}{X7CEBD97183C3399E}{}
{\noindent\textcolor{FuncColor}{$\triangleright$\enspace\texttt{ConjugacyClassRepsCompatibleSubgroups({\mdseries\slshape F})\index{ConjugacyClassRepsCompatibleSubgroups@\texttt{Conjugacy}\-\texttt{Class}\-\texttt{Reps}\-\texttt{Compatible}\-\texttt{Subgroups}!for IsLocalAction}
\label{ConjugacyClassRepsCompatibleSubgroups:for IsLocalAction}
}\hfill{\scriptsize (attribute)}}\\
\textbf{\indent Returns:\ }
a list of compatible representatives of conjugacy classes of \mbox{\texttt{\mdseries\slshape F}} that contain a compatible subgroup. 



 The argument of this method is a local action \mbox{\texttt{\mdseries\slshape F}} of $\mathrm{Aut}(B_{d,k})$. 

 }

 

 
\begin{Verbatim}[commandchars=!@|,fontsize=\small,frame=single,label=Example]
  !gapprompt@gap>| !gapinput@ConjugacyClassRepsCompatibleSubgroups(3,2,AutB(3,2));|
  [ Group(()), Group([ (1,2)(3,5)(4,6) ]), Group([ (1,4,5)(2,3,6) ]), 
    Group([ (3,5)(4,6), (1,2) ]), Group([ (1,2)(3,5)(4,6), (1,3,6)
    (2,4,5) ]), Group([ (3,5)(4,6), (1,3,5)(2,4,6), (1,2)(3,4)(5,6) ]), 
    Group([ (1,2)(3,5)(4,6), (1,3,5)(2,4,6), (1,2)(5,6), (1,2)(3,4) ]), 
    Group([ (3,5)(4,6), (1,3,5)(2,4,6), (1,2)(5,6), (1,2)(3,4) ]), 
    Group([ (5,6), (3,4), (1,2), (1,3,5)(2,4,6), (3,5)(4,6) ]) ]
\end{Verbatim}
 

\subsection{\textcolor{Chapter }{ConjugacyClassRepsCompatibleSubgroupsWithProjection}}
\logpage{[ 3, 3, 5 ]}\nobreak
\hyperdef{L}{X876408A9784A6F34}{}
{\noindent\textcolor{FuncColor}{$\triangleright$\enspace\texttt{ConjugacyClassRepsCompatibleSubgroupsWithProjection({\mdseries\slshape l, F})\index{ConjugacyClassRepsCompatibleSubgroupsWithProjection@\texttt{Conjugacy}\-\texttt{Class}\-\texttt{Reps}\-\texttt{Compatible}\-\texttt{Subgroups}\-\texttt{With}\-\texttt{Projection}}
\label{ConjugacyClassRepsCompatibleSubgroupsWithProjection}
}\hfill{\scriptsize (function)}}\\
\textbf{\indent Returns:\ }
 a list of compatible representatives of conjugacy classes of $\mathrm{Aut}(B_{d,k})$ that contain a compatible subgroup which projects to \mbox{\texttt{\mdseries\slshape F}} $\le\mathrm{Aut}(B_{d,r})$. 



 The arguments of this method are a radius \mbox{\texttt{\mdseries\slshape r}} $\in\mathbb{N}$, and a local action \mbox{\texttt{\mdseries\slshape F}} $\le\mathrm{Aut}(B_{d,k})$ for some $k\le r$. 

 }

 

 
\begin{Verbatim}[commandchars=!@|,fontsize=\small,frame=single,label=Example]
  !gapprompt@gap>| !gapinput@S3:=SymmetricGroup(3);;|
  !gapprompt@gap>| !gapinput@ConjugacyClassRepsCompatibleSubgroupsWithProjection(3,2,1,S3);|
  [ Group([ (1,2)(3,5)(4,6), (1,4,5)(2,3,6) ]), Group([ (1,2)(3,4)
    (5,6), (1,2)(3,5)(4,6), (1,4,5)(2,3,6) ]), Group([ (3,4)(5,6), (1,2)
    (3,4), (1,4,5)(2,3,6), (3,5,4,6) ]), Group([ (3,4)(5,6), (1,2)
    (3,4), (1,4,5)(2,3,6), (3,5)(4,6) ]), Group([ (3,4)(5,6), (1,2)
    (3,4), (1,4,5)(2,3,6), (5,6), (3,5,4,6) ]) ]
  !gapprompt@gap>| !gapinput@A3:=AlternatingGroup(3);;|
  !gapprompt@gap>| !gapinput@ConjugacyClassRepsCompatibleSubgroupsWithProjection(3,2,1,A3);|
  [ Group([ (1,4,5)(2,3,6) ]) ]
\end{Verbatim}
 

 
\begin{Verbatim}[commandchars=!@|,fontsize=\small,frame=single,label=Example]
  !gapprompt@gap>| !gapinput@F:=SymmetricGroup(3);;|
  !gapprompt@gap>| !gapinput@rho:=SignHomomorphism(F);;|
  !gapprompt@gap>| !gapinput@H1:=PI(2,3,F,rho,[0,1]);;|
  !gapprompt@gap>| !gapinput@H2:=PI(2,3,F,rho,[1]);;|
  !gapprompt@gap>| !gapinput@Size(ConjugacyClassRepsCompatibleSubgroupsWithProjection(3,3,2,H1));|
  2
  !gapprompt@gap>| !gapinput@Size(ConjugacyClassRepsCompatibleSubgroupsWithProjection(3,3,2,H2));|
  4
\end{Verbatim}
 }

 }

   
\chapter{\textcolor{Chapter }{Examples}}\label{Chapter_Examples}
\logpage{[ 4, 0, 0 ]}
\hyperdef{L}{X7A489A5D79DA9E5C}{}
{
  

 Several classes of examples of subgroups of $\mathrm{Aut}(B_{d,k})$ that satisfy (C) and or (D) are constructed in \cite{Tor20} and implemented in this section. For a given permutation group $F\le S_{d}$, there are always the three local actions $\Gamma(F)$, $\Delta(F)$ and $\Phi(F)$ on $\mathrm{Aut}(B_{d,2})$ that project onto $F$. For some $F$, these are all distinct and yield all universal groups that have $F$ as their $1$-local action, see \cite[Theorem 3.32]{Tor20}. More examples arise in particular when either point stabilizers in $F$ are not simple, $F$ preserves a partition, or $F$ is not perfect. 
\section{\textcolor{Chapter }{Discrete groups}}\label{Chapter_Examples_Section_Discrete_groups}
\logpage{[ 4, 1, 0 ]}
\hyperdef{L}{X7A91225C7FCEBA7D}{}
{
  Here, we implement the local actions $\Gamma(F),\Delta(F)\le\mathrm{Aut}(B_{d,2})$, both of which satisfy both (C) and (D), see \cite[Section 3.4.1]{Tor20}. 

 
\subsection{\textcolor{Chapter }{gamma}}\label{gamma}
\logpage{[ 4, 1, 1 ]}
\hyperdef{L}{X87E33E187FE7341A}{}
{
\noindent\textcolor{FuncColor}{$\triangleright$\enspace\texttt{gamma({\mdseries\slshape d, a})\index{gamma@\texttt{gamma}!for d, a}
\label{gamma:for d, a}
}\hfill{\scriptsize (operation)}}\\
\noindent\textcolor{FuncColor}{$\triangleright$\enspace\texttt{gamma({\mdseries\slshape l, d, a})\index{gamma@\texttt{gamma}!for l, d, a}
\label{gamma:for l, d, a}
}\hfill{\scriptsize (operation)}}\\
\noindent\textcolor{FuncColor}{$\triangleright$\enspace\texttt{gamma({\mdseries\slshape l, d, s, addr})\index{gamma@\texttt{gamma}!for l, d, s, addr}
\label{gamma:for l, d, s, addr}
}\hfill{\scriptsize (operation)}}\\
\noindent\textcolor{FuncColor}{$\triangleright$\enspace\texttt{gamma({\mdseries\slshape d, k, aut, z})\index{gamma@\texttt{gamma}!for d, k, aut, z}
\label{gamma:for d, k, aut, z}
}\hfill{\scriptsize (operation)}}\\


 

 
\begin{description}
\item[{for the arguments \mbox{\texttt{\mdseries\slshape d}}, \mbox{\texttt{\mdseries\slshape a}}}]  Returns: the automorphism $\gamma($\mbox{\texttt{\mdseries\slshape a}}$)=($\mbox{\texttt{\mdseries\slshape a}}$,($\mbox{\texttt{\mdseries\slshape a}}$)_{\omega\in\Omega})\in\mathrm{Aut}(B_{d,2})$. 

 The arguments of this method are a degree \mbox{\texttt{\mdseries\slshape d}} $\in\mathbb{N}_{\ge 3}$ and a permutation \mbox{\texttt{\mdseries\slshape a}} $\in S_d$. 
\item[{for the arguments \mbox{\texttt{\mdseries\slshape l}}, \mbox{\texttt{\mdseries\slshape d}}, \mbox{\texttt{\mdseries\slshape a}}}]  Returns: the automorphism $\gamma^{l}($\mbox{\texttt{\mdseries\slshape a}}$)\in\mathrm{Aut}(B_{d,l})$ all of whose $1$-local actions are given by \mbox{\texttt{\mdseries\slshape a}}. 

 The arguments of this method are a radius \mbox{\texttt{\mdseries\slshape l}} $\in\mathbb{N}$, a degree \mbox{\texttt{\mdseries\slshape d}} $\in\mathbb{N}_{\ge 3}$ and a permutation \mbox{\texttt{\mdseries\slshape a}} $\in S_d$. 
\item[{for the arguments \mbox{\texttt{\mdseries\slshape l}}, \mbox{\texttt{\mdseries\slshape d}}, \mbox{\texttt{\mdseries\slshape s}}, \mbox{\texttt{\mdseries\slshape addr}}}]  Returns: the automorphism of $B_{d,k}$ whose $1$-local actions are given by \mbox{\texttt{\mdseries\slshape s}} at vertices whose address has \mbox{\texttt{\mdseries\slshape addr}} as a prefix and are trivial elsewhere. 

 The arguments of this method are a radius \mbox{\texttt{\mdseries\slshape l}} $\in\mathbb{N}$, a degree \mbox{\texttt{\mdseries\slshape d}} $\in\mathbb{N}_{\ge 3}$, a permutation \mbox{\texttt{\mdseries\slshape s}} $\in S_d$ and an address \mbox{\texttt{\mdseries\slshape addr}} of a vertex in $B_{d,l}$ whose last entry is fixed by \mbox{\texttt{\mdseries\slshape s}}. 
\item[{for the arguments \mbox{\texttt{\mdseries\slshape d}}, \mbox{\texttt{\mdseries\slshape k}}, \mbox{\texttt{\mdseries\slshape aut}}, \mbox{\texttt{\mdseries\slshape z}}}]  Returns: the automorphism $\gamma_{z}($\mbox{\texttt{\mdseries\slshape aut}}$)=($\mbox{\texttt{\mdseries\slshape aut}}$,($\mbox{\texttt{\mdseries\slshape z}}$($\mbox{\texttt{\mdseries\slshape aut}}$,\omega))_{\omega\in\Omega})\in\mathrm{Aut}(B_{d,k+1})$. 

 The arguments of this method are a degree \mbox{\texttt{\mdseries\slshape d}} $\in\mathbb{N}_{\ge 3}$, a radius \mbox{\texttt{\mdseries\slshape k}} $\in\mathbb{N}$, an automorphism \mbox{\texttt{\mdseries\slshape aut}} of $B_{d,k}$, and an involutive compatibility cocycle \mbox{\texttt{\mdseries\slshape z}} of a subgroup of $\mathrm{Aut}(B_{d,k})$ that contains \mbox{\texttt{\mdseries\slshape aut}} (see \texttt{InvolutiveCompatibilityCocycle} (\ref{InvolutiveCompatibilityCocycle})). 
\end{description}
 

 }

 

 
\begin{Verbatim}[commandchars=!@|,fontsize=\small,frame=single,label=Example]
  !gapprompt@gap>| !gapinput@gamma(3,(1,2));|
  (1,3)(2,4)(5,6)
\end{Verbatim}
 

 
\begin{Verbatim}[commandchars=!@|,fontsize=\small,frame=single,label=Example]
  !gapprompt@gap>| !gapinput@gamma(2,3,(1,2));|
  (1,3)(2,4)(5,6)
  !gapprompt@gap>| !gapinput@gamma(3,3,(1,2));|
  (1,5)(2,6)(3,8)(4,7)(9,11)(10,12)
\end{Verbatim}
 

 
\begin{Verbatim}[commandchars=!@|,fontsize=\small,frame=single,label=Example]
  !gapprompt@gap>| !gapinput@gamma(3,3,(1,2),[1,3]);|
  (3,4)
  !gapprompt@gap>| !gapinput@gamma(3,3,(1,2),[]);|
  (1,5)(2,6)(3,8)(4,7)(9,11)(10,12)
\end{Verbatim}
 

 
\begin{Verbatim}[commandchars=!@|,fontsize=\small,frame=single,label=Example]
  !gapprompt@gap>| !gapinput@S3:=SymmetricGroup(3);;|
  !gapprompt@gap>| !gapinput@z1:=AllInvolutiveCompatibilityCocycles(3,1,S3)[1];;|
  !gapprompt@gap>| !gapinput@gamma(3,1,(1,2),z1);|
  (1,4)(2,3)(5,6)
  !gapprompt@gap>| !gapinput@z3:=AllInvolutiveCompatibilityCocycles(3,1,S3)[3];;|
  !gapprompt@gap>| !gapinput@gamma(3,1,(1,2),z3);|
  (1,3)(2,4)(5,6)
\end{Verbatim}
 

 
\subsection{\textcolor{Chapter }{GAMMA}}\label{GAMMA}
\logpage{[ 4, 1, 2 ]}
\hyperdef{L}{X87E33E187FE7341A}{}
{
\noindent\textcolor{FuncColor}{$\triangleright$\enspace\texttt{GAMMA({\mdseries\slshape d, F})\index{GAMMA@\texttt{GAMMA}!for d, F}
\label{GAMMA:for d, F}
}\hfill{\scriptsize (operation)}}\\
\noindent\textcolor{FuncColor}{$\triangleright$\enspace\texttt{GAMMA({\mdseries\slshape l, d, F})\index{GAMMA@\texttt{GAMMA}!for l, d, F}
\label{GAMMA:for l, d, F}
}\hfill{\scriptsize (operation)}}\\
\noindent\textcolor{FuncColor}{$\triangleright$\enspace\texttt{GAMMA({\mdseries\slshape F, z})\index{GAMMA@\texttt{GAMMA}!for F, z}
\label{GAMMA:for F, z}
}\hfill{\scriptsize (operation)}}\\


 
\begin{description}
\item[{for the arguments \mbox{\texttt{\mdseries\slshape d}}, \mbox{\texttt{\mdseries\slshape F}}}]  Returns: the local action $\Gamma($\mbox{\texttt{\mdseries\slshape F}}$)=\{(a,(a)_{\omega})\mid a\in F\}\le\mathrm{Aut}(B_{d,2})$. 

 The arguments of this method are a degree \mbox{\texttt{\mdseries\slshape d}} $\in\mathbb{N}_{\ge 3}$, and a subgroup \mbox{\texttt{\mdseries\slshape F}} of $S_{d}$. 
\item[{for the arguments \mbox{\texttt{\mdseries\slshape l}}, \mbox{\texttt{\mdseries\slshape d}}, \mbox{\texttt{\mdseries\slshape F}}}]  Returns: the group $\Gamma^{l}($\mbox{\texttt{\mdseries\slshape F}}$)\le\mathrm{Aut}(B_{d,l})$. 

 The arguments of this method are a radius \mbox{\texttt{\mdseries\slshape l}} $\in\mathbb{N}$, a degree \mbox{\texttt{\mdseries\slshape d}} $\in\mathbb{N}_{\ge 3}$, and a subgroup \mbox{\texttt{\mdseries\slshape F}} of $S_d$. 
\item[{for the arguments \mbox{\texttt{\mdseries\slshape d}}, \mbox{\texttt{\mdseries\slshape k}}, \mbox{\texttt{\mdseries\slshape F}}, \mbox{\texttt{\mdseries\slshape z}}}]  Returns: the group $\Gamma_{z}($\mbox{\texttt{\mdseries\slshape F}}$)=\{(a,($\mbox{\texttt{\mdseries\slshape z}}$(a,\omega))_{\omega\in\Omega})\mid a\in$\mbox{\texttt{\mdseries\slshape F}}$\}\le\mathrm{Aut}(B_{d,k+1})$. 

 The arguments of this method are a local action \mbox{\texttt{\mdseries\slshape F}} $\le\mathrm{Aut}(B_{d,k})$ and an involutive compatibility cocycle \mbox{\texttt{\mdseries\slshape z}} of \mbox{\texttt{\mdseries\slshape F}} (see \texttt{InvolutiveCompatibilityCocycle} (\ref{InvolutiveCompatibilityCocycle})). 
\end{description}
 

 }

 

 
\begin{Verbatim}[commandchars=!@|,fontsize=\small,frame=single,label=Example]
  !gapprompt@gap>| !gapinput@F:=TransitiveGroup(4,3);;|
  !gapprompt@gap>| !gapinput@GAMMA(4,F);|
  Group([ (1,5,9,10)(2,6,7,11)(3,4,8,12), (1,8)(2,7)(3,9)(4,5)(10,12) ])
\end{Verbatim}
 

 
\begin{Verbatim}[commandchars=!@|,fontsize=\small,frame=single,label=Example]
  !gapprompt@gap>| !gapinput@GAMMA(3,SymmetricGroup(3));|
  Group([ (1,4,5)(2,3,6), (1,3)(2,4)(5,6) ])
  !gapprompt@gap>| !gapinput@GAMMA(2,3,SymmetricGroup(3));|
  Group([ (1,4,5)(2,3,6), (1,3)(2,4)(5,6) ])
  !gapprompt@gap>| !gapinput@GAMMA(3,3,SymmetricGroup(3));|
  Group([ (1,8,10)(2,7,9)(3,5,12)(4,6,11), (1,5)(2,6)(3,8)(4,7)(9,11)(10,12) ])
\end{Verbatim}
 

 
\begin{Verbatim}[commandchars=!@|,fontsize=\small,frame=single,label=Example]
  !gapprompt@gap>| !gapinput@F:=SymmetricGroup(3);;|
  !gapprompt@gap>| !gapinput@rho:=SignHomomorphism(F);;|
  !gapprompt@gap>| !gapinput@H:=PI(2,3,F,rho,[1]);;|
  !gapprompt@gap>| !gapinput@z:=InvolutiveCompatibilityCocycle(3,2,H);;|
  !gapprompt@gap>| !gapinput@GAMMA(3,2,H,z);|
  Group([ (), (), (1,9)(2,10)(3,12)(4,11)(7,8), (1,10,3,11)(2,9,4,12)
    (5,8,6,7), (1,12,8)(2,11,7)(3,10,5)(4,9,6) ])
\end{Verbatim}
 
\subsection{\textcolor{Chapter }{DELTA}}\label{DELTA}
\logpage{[ 4, 1, 3 ]}
\hyperdef{L}{X82C33CF282FC5A73}{}
{
\noindent\textcolor{FuncColor}{$\triangleright$\enspace\texttt{DELTA({\mdseries\slshape d, F})\index{DELTA@\texttt{DELTA}!for d, F}
\label{DELTA:for d, F}
}\hfill{\scriptsize (operation)}}\\
\noindent\textcolor{FuncColor}{$\triangleright$\enspace\texttt{DELTA({\mdseries\slshape d, F, C})\index{DELTA@\texttt{DELTA}!for d, F, C}
\label{DELTA:for d, F, C}
}\hfill{\scriptsize (operation)}}\\


 

 
\begin{description}
\item[{for the arguments \mbox{\texttt{\mdseries\slshape d}}, \mbox{\texttt{\mdseries\slshape F}}}]  Returns: the group $\Delta($\mbox{\texttt{\mdseries\slshape F}}$)\le\mathrm{Aut}(B_{d,2})$. 

 The arguments of this method are a degree \mbox{\texttt{\mdseries\slshape d}} $\in\mathbb{N}_{\ge 3}$, and a \emph{transitive} subgroup \mbox{\texttt{\mdseries\slshape F}} of $S_{d}$. 
\item[{for the arguments \mbox{\texttt{\mdseries\slshape d}}, \mbox{\texttt{\mdseries\slshape F}}, \mbox{\texttt{\mdseries\slshape C}}}]  Returns: the group $\Delta($\mbox{\texttt{\mdseries\slshape F}}$,$\mbox{\texttt{\mdseries\slshape C}}$)\le\mathrm{Aut}(B_{d,2})$. 

 The arguments of this method are a degree \mbox{\texttt{\mdseries\slshape d}} $\in\mathbb{N}_{\ge 3}$, a \emph{transitive} subgroup \mbox{\texttt{\mdseries\slshape F}} of $S_d$, and a central subgroup \mbox{\texttt{\mdseries\slshape C}} of the stabilizer \mbox{\texttt{\mdseries\slshape F}}$_{1}$ of $1$ in \mbox{\texttt{\mdseries\slshape F}}. 
\end{description}
 

 }

 

 
\begin{Verbatim}[commandchars=!@|,fontsize=\small,frame=single,label=Example]
  !gapprompt@gap>| !gapinput@F:=SymmetricGroup(3);;|
  !gapprompt@gap>| !gapinput@D:=DELTA(3,F);|
  Group([ (1,3,6)(2,4,5), (1,3)(2,4), (1,2)(3,4)(5,6) ])
  !gapprompt@gap>| !gapinput@F1:=Stabilizer(F,1);;|
  !gapprompt@gap>| !gapinput@D1:=DELTA(3,F,F1);|
  Group([ (1,4,5)(2,3,6), (1,3)(2,4)(5,6), (1,2)(3,4)(5,6) ])
  !gapprompt@gap>| !gapinput@D=D1;|
  false
  !gapprompt@gap>| !gapinput@G:=AutB(3,2);;|
  !gapprompt@gap>| !gapinput@D^G=D1^G;|
  true
\end{Verbatim}
 

 
\begin{Verbatim}[commandchars=!@|,fontsize=\small,frame=single,label=Example]
  !gapprompt@gap>| !gapinput@F:=PrimitiveGroup(5,3);|
  AGL(1, 5)
  !gapprompt@gap>| !gapinput@F1:=Stabilizer(F,1);|
  Group([ (2,3,4,5) ])
  !gapprompt@gap>| !gapinput@C:=Group((2,4)(3,5));|
  Group([ (2,4)(3,5) ])
  !gapprompt@gap>| !gapinput@Index(F1,C);|
  2
  !gapprompt@gap>| !gapinput@Index(DELTA(5,F,F1),DELTA(5,F,C));|
  2
\end{Verbatim}
 

 }

 
\section{\textcolor{Chapter }{Maximal extensions}}\label{Chapter_Examples_Section_Maximal_extensions}
\logpage{[ 4, 2, 0 ]}
\hyperdef{L}{X7F731157843279EC}{}
{
  For any $F\le\mathrm{Aut}(B_{d,k})$ that satisfies (C), the group $\Phi(F)\le\mathrm{Aut}(B_{d,k+1})$ is the maximal extension of $F$ that satisfies (C) as well. It stems from the action of $\mathrm{U}_{k}(F)$ on balls of radius $k+1$ in $T_{d}$. 

 
\subsection{\textcolor{Chapter }{PHI}}\label{PHI1}
\logpage{[ 4, 2, 1 ]}
\hyperdef{L}{X85A0C67982D9057A}{}
{
\noindent\textcolor{FuncColor}{$\triangleright$\enspace\texttt{PHI({\mdseries\slshape d, F})\index{PHI@\texttt{PHI}!for d, F}
\label{PHI:for d, F}
}\hfill{\scriptsize (operation)}}\\
\noindent\textcolor{FuncColor}{$\triangleright$\enspace\texttt{PHI({\mdseries\slshape F})\index{PHI@\texttt{PHI}!for F}
\label{PHI:for F}
}\hfill{\scriptsize (operation)}}\\
\noindent\textcolor{FuncColor}{$\triangleright$\enspace\texttt{PHI({\mdseries\slshape l, F})\index{PHI@\texttt{PHI}!for l, F}
\label{PHI:for l, F}
}\hfill{\scriptsize (operation)}}\\


 
\begin{description}
\item[{for the arguments \mbox{\texttt{\mdseries\slshape d}}, \mbox{\texttt{\mdseries\slshape F}}}]  Returns: the group $\Phi($\mbox{\texttt{\mdseries\slshape F}}$)=\{(a,(a_{\omega})_{\omega})\mid a\in $\mbox{\texttt{\mdseries\slshape F}}$,\ \forall \omega\in\Omega:\ a_{\omega}\in
C_{F}(a,\omega)\}\le\mathrm{Aut}(B_{d,2})$. 

 The arguments of this method are a degree \mbox{\texttt{\mdseries\slshape d}} $\in\mathbb{N}_{\ge 3}$ and a permutation group \mbox{\texttt{\mdseries\slshape F}} $\le S_{d}$. 
\item[{for the arguments \mbox{\texttt{\mdseries\slshape d}}, \mbox{\texttt{\mdseries\slshape k}}, \mbox{\texttt{\mdseries\slshape F}}}]  Returns: the group $\Phi_{k}($\mbox{\texttt{\mdseries\slshape F}}$)=\{(a,(a_{\omega})_{\omega})\mid a\in $\mbox{\texttt{\mdseries\slshape F}}$,\ \forall \omega\in\Omega:\ a_{\omega}\in
C_{F}(a,\omega)\}\le\mathrm{Aut}(B_{d,k+1})$. 

 The argument of this method is a local action \mbox{\texttt{\mdseries\slshape F}} $\le\mathrm{Aut}(B_{d,k})$. 
\item[{for the arguments \mbox{\texttt{\mdseries\slshape l}}, \mbox{\texttt{\mdseries\slshape d}}, \mbox{\texttt{\mdseries\slshape k}}, \mbox{\texttt{\mdseries\slshape F}}}]  Returns: the group $\Phi^{l}($\mbox{\texttt{\mdseries\slshape F}}$)=\Phi_{l-1}\circ\cdots\circ\Phi_{k+1}\circ\Phi_{k}($\mbox{\texttt{\mdseries\slshape F}}$)\le\mathrm{Aut}(B_{d,l})$. 

 The arguments of this method are a radius \mbox{\texttt{\mdseries\slshape l}} $\in\mathbb{N}$ and a local action \mbox{\texttt{\mdseries\slshape F}} $\le\mathrm{Aut}(B_{d,k})$. 
\end{description}
 

 }

 

 
\begin{Verbatim}[commandchars=!@|,fontsize=\small,frame=single,label=Example]
  !gapprompt@gap>| !gapinput@PHI(3,SymmetricGroup(3));|
  Group([ (1,4,5)(2,3,6), (1,3)(2,4)(5,6), (1,2), (3,4), (5,6) ])
  !gapprompt@gap>| !gapinput@last=AutB(3,2);|
  true
  !gapprompt@gap>| !gapinput@PHI(3,AlternatingGroup(3));|
  Group([ (1,4,5)(2,3,6) ])
  !gapprompt@gap>| !gapinput@last=GAMMA(3,AlternatingGroup(3));|
  true
\end{Verbatim}
 

 
\begin{Verbatim}[commandchars=!@|,fontsize=\small,frame=single,label=Example]
  !gapprompt@gap>| !gapinput@S3:=SymmetricGroup(3);;|
  !gapprompt@gap>| !gapinput@groups:=ConjugacyClassRepsCompatibleSubgroupsWithProjection(3,2,1,S3);|
  [ Group([ (1,2)(3,5)(4,6), (1,4,5)(2,3,6) ]), 
    Group([ (1,2)(3,4)(5,6), (1,2)(3,5)(4,6), (1,4,5)(2,3,6) ]), 
    Group([ (3,4)(5,6), (1,2)(3,4), (1,4,5)(2,3,6), (3,5,4,6) ]), 
    Group([ (3,4)(5,6), (1,2)(3,4), (1,4,5)(2,3,6), (3,5)(4,6) ]), 
    Group([ (3,4)(5,6), (1,2)(3,4), (1,4,5)(2,3,6), (5,6), (3,5,4,6) ]) ]
  !gapprompt@gap>| !gapinput@for G in groups do Print(Size(G),",",Size(PHI(3,2,G)),"\n"); od;|
  6,6
  12,12
  24,192
  24,192
  48,3072
\end{Verbatim}
 

 
\begin{Verbatim}[commandchars=!@|,fontsize=\small,frame=single,label=Example]
  !gapprompt@gap>| !gapinput@PHI(3,4,1,SymmetricGroup(4));|
  <permutation group with 34 generators>
  !gapprompt@gap>| !gapinput@last=AutB(4,3);|
  true
\end{Verbatim}
 

 
\begin{Verbatim}[commandchars=!@|,fontsize=\small,frame=single,label=Example]
  !gapprompt@gap>| !gapinput@rho:=SignHomomorphism(SymmetricGroup(3));;|
  !gapprompt@gap>| !gapinput@F:=PI(2,3,SymmetricGroup(3),rho,[1]);; Size(F);|
  24
  !gapprompt@gap>| !gapinput@P:=PHI(4,3,2,F);; Size(P);|
  12288
  !gapprompt@gap>| !gapinput@IsSubgroup(AutB(3,4),P);|
  true
  !gapprompt@gap>| !gapinput@IsCompatible(3,4,P);|
  true
\end{Verbatim}
 

 }

 
\section{\textcolor{Chapter }{Normal subgroups and partitions}}\label{Chapter_Examples_Section_Normal_subgroups_and_partitions}
\logpage{[ 4, 3, 0 ]}
\hyperdef{L}{X80EDDA597A66421E}{}
{
  When point stabilizers in $F\le S_{d}$ are not simple, or $F$ preserves a partition, more universal groups can be constructed as follows. 

 
\subsection{\textcolor{Chapter }{PHI}}\label{PHI2}
\logpage{[ 4, 3, 1 ]}
\hyperdef{L}{X85A0C67982D9057A}{}
{
\noindent\textcolor{FuncColor}{$\triangleright$\enspace\texttt{PHI({\mdseries\slshape d, F, N})\index{PHI@\texttt{PHI}!for d, F, N}
\label{PHI:for d, F, N}
}\hfill{\scriptsize (operation)}}\\
\noindent\textcolor{FuncColor}{$\triangleright$\enspace\texttt{PHI({\mdseries\slshape d, F, P})\index{PHI@\texttt{PHI}!for d, F, P}
\label{PHI:for d, F, P}
}\hfill{\scriptsize (operation)}}\\
\noindent\textcolor{FuncColor}{$\triangleright$\enspace\texttt{PHI({\mdseries\slshape F, P})\index{PHI@\texttt{PHI}!for F, P}
\label{PHI:for F, P}
}\hfill{\scriptsize (operation)}}\\


 

 
\begin{description}
\item[{for the arguments \mbox{\texttt{\mdseries\slshape d}}, \mbox{\texttt{\mdseries\slshape F}}, \mbox{\texttt{\mdseries\slshape N}}}]  Returns: the group $\Phi($\mbox{\texttt{\mdseries\slshape F}}$,$\mbox{\texttt{\mdseries\slshape N}}$)\le\mathrm{Aut}(B_{d,2})$. 

 The arguments of this method are a degree \mbox{\texttt{\mdseries\slshape d}} $\in\mathbb{N}_{\ge 3}$, a \emph{transitive} permutation group \mbox{\texttt{\mdseries\slshape F}} $\le S_{d}$ and a normal subgroup \mbox{\texttt{\mdseries\slshape N}} of the stabilizer \mbox{\texttt{\mdseries\slshape F}}$_{1}$ of $1$ in \mbox{\texttt{\mdseries\slshape F}}. 
\item[{for the arguments \mbox{\texttt{\mdseries\slshape d}}, \mbox{\texttt{\mdseries\slshape F}}, \mbox{\texttt{\mdseries\slshape P}}}]  Returns: the group $\Phi($\mbox{\texttt{\mdseries\slshape F}}$,$\mbox{\texttt{\mdseries\slshape P}}$)=\{(a,(a_{\omega})_{\omega})\mid a\in $\mbox{\texttt{\mdseries\slshape F}}$,\ a_{\omega}\in C_{F}(a,\omega)$ constant w.r.t. \mbox{\texttt{\mdseries\slshape P}}$\}\le\mathrm{Aut}(B_{d,2})$. 

 The arguments of this method are a degree \mbox{\texttt{\mdseries\slshape d}} $\in\mathbb{N}_{\ge 3}$ and a permutation group \mbox{\texttt{\mdseries\slshape F}} $\le S_{d}$ and a partition \mbox{\texttt{\mdseries\slshape P}} of \texttt{[1..d]} preserved by \mbox{\texttt{\mdseries\slshape F}}. 
\item[{for the arguments \mbox{\texttt{\mdseries\slshape d}}, \mbox{\texttt{\mdseries\slshape k}}, \mbox{\texttt{\mdseries\slshape F}}, \mbox{\texttt{\mdseries\slshape P}}}]  Returns: the group $\Phi_{k}($\mbox{\texttt{\mdseries\slshape F}}$,$\mbox{\texttt{\mdseries\slshape P}}$)=\{(\alpha,(\alpha_{\omega})_{\omega})\mid \alpha\in \mbox{\texttt{\mdseries\slshape F}},\ \alpha_{\omega}\in C_{F}(\alpha,\omega)$ constant w.r.t. \mbox{\texttt{\mdseries\slshape P}}$\}\le\mathrm{Aut}(B_{d,k+1})$. 

 The arguments of this method are a local action \mbox{\texttt{\mdseries\slshape F}} $\le\mathrm{Aut}(B_{d,k})$ and a partition \mbox{\texttt{\mdseries\slshape P}} of \texttt{[1..d]} preserverd by $\pi$\mbox{\texttt{\mdseries\slshape F}} $\le S_{d}$. This method assumes that all compatibility sets with respect to a partition
element are non-empty and that all compatibility sets of the identity with
respect to a partition element are non-trivial. 
\end{description}
 

 }

 

 
\begin{Verbatim}[commandchars=!@|,fontsize=\small,frame=single,label=Example]
  !gapprompt@gap>| !gapinput@F:=SymmetricGroup(4);;|
  !gapprompt@gap>| !gapinput@F1:=Stabilizer(P,1);|
  Sym( [ 2 .. 4 ] )
  !gapprompt@gap>| !gapinput@grps:=NormalSubgroups(F1);|
  [ Sym( [ 2 .. 4 ] ), Alt( [ 2 .. 4 ] ), Group(()) ]
  !gapprompt@gap>| !gapinput@N:=grps[2];|
  Alt( [ 2 .. 4 ] )
  !gapprompt@gap>| !gapinput@PHI(4,F,N);|
  Group([ (1,5,9,10)(2,6,7,11)(3,4,8,12), (1,4)(2,5)(3,6)(7,8)(10,11), 
    (1,2,3) ])
  !gapprompt@gap>| !gapinput@Index(F1,N);|
  2
  !gapprompt@gap>| !gapinput@Index(PHI(4,F,F1),PHI(4,F,N));|
  16
\end{Verbatim}
 

 
\begin{Verbatim}[commandchars=!@|,fontsize=\small,frame=single,label=Example]
  !gapprompt@gap>| !gapinput@F:=TransitiveGroup(4,3);|
  D(4)
  !gapprompt@gap>| !gapinput@P:=Blocks(F,[1..4]);|
  [ [ 1, 3 ], [ 2, 4 ] ]
  !gapprompt@gap>| !gapinput@G:=PHI(4,F,P);|
  Group([ (1,5,9,10)(2,6,7,11)(3,4,8,12), (1,8)(2,7)(3,9)(4,5)(10,12), (1,3)
    (8,9), (4,5)(10,12) ])
  !gapprompt@gap>| !gapinput@aut:=Random(G);|
  (1,5,9,10)(2,6,7,11)(3,4,8,12)
  !gapprompt@gap>| !gapinput@LocalAction(1,4,2,a,[1]); LocalAction(1,4,2,a,[3]);|
  (1,2,3,4)
  (1,2,3,4)
  !gapprompt@gap>| !gapinput@LocalAction(1,4,2,a,[2]); LocalAction(1,4,2,a,[4]);|
  (1,4)(2,3)
  (1,4)(2,3)
\end{Verbatim}
 

 
\begin{Verbatim}[commandchars=!@|,fontsize=\small,frame=single,label=Example]
  !gapprompt@gap>| !gapinput@H:=TransitiveGroup(4,3);|
  D(4)
  !gapprompt@gap>| !gapinput@P:=Blocks(H,[1..4]);|
  [ [ 1, 3 ], [ 2, 4 ] ]
  !gapprompt@gap>| !gapinput@F:=PHI(4,H,P);;|
  !gapprompt@gap>| !gapinput@G:=PHI(4,2,F,P);|
  <permutation group with 5 generators>
  !gapprompt@gap>| !gapinput@IsCompatible(4,3,G);|
  true
\end{Verbatim}
 

 }

 
\section{\textcolor{Chapter }{Abelian quotients}}\label{Chapter_Examples_Section_Abelian_quotients}
\logpage{[ 4, 4, 0 ]}
\hyperdef{L}{X784DFD007FE78DE9}{}
{
  When a permutation group $F\le S_{d}$ is not perfect, i.e. it admits an abelian quotient $\rho:F\twoheadrightarrow A$, more universal groups can be constructed by imposing restrictions of the
form $\prod_{r\in R}\prod_{x\in S(b,r)}\rho(\sigma_{1}(\alpha,x))=1$ on elements $\alpha\in\Phi^{k}(F)\le\mathrm{Aut}(B_{d,k})$. 

 

\subsection{\textcolor{Chapter }{SignHomomorphism}}
\logpage{[ 4, 4, 1 ]}\nobreak
\hyperdef{L}{X7F4CC068860BA22F}{}
{\noindent\textcolor{FuncColor}{$\triangleright$\enspace\texttt{SignHomomorphism({\mdseries\slshape F})\index{SignHomomorphism@\texttt{SignHomomorphism}}
\label{SignHomomorphism}
}\hfill{\scriptsize (function)}}\\
\textbf{\indent Returns:\ }
 the sign homomorphism from \mbox{\texttt{\mdseries\slshape F}} to $S_{2}$. 



 The argument of this method is a permutation group \mbox{\texttt{\mdseries\slshape F}} $\le S_{d}$. This method can be used as an example for the argument \mbox{\texttt{\mdseries\slshape rho}} in the methods \texttt{SpheresProduct} (\ref{SpheresProduct}) and \texttt{PI} (\ref{PI}). 

 }

 

 
\begin{Verbatim}[commandchars=!@|,fontsize=\small,frame=single,label=Example]
  !gapprompt@gap>| !gapinput@F:=SymmetricGroup(3);;|
  !gapprompt@gap>| !gapinput@sign:=SignHomomorphism(F);|
  MappingByFunction( Sym( [ 1 .. 3 ] ), Sym( [ 1 .. 2 ] ), function( g ) ... end )
  !gapprompt@gap>| !gapinput@Image(sign,(2,3));|
  (1,2)
  !gapprompt@gap>| !gapinput@Image(sign,(1,2,3));|
  ()
\end{Verbatim}
 

\subsection{\textcolor{Chapter }{AbelianizationHomomorphism}}
\logpage{[ 4, 4, 2 ]}\nobreak
\hyperdef{L}{X7D8D1FC07BC90971}{}
{\noindent\textcolor{FuncColor}{$\triangleright$\enspace\texttt{AbelianizationHomomorphism({\mdseries\slshape F})\index{AbelianizationHomomorphism@\texttt{AbelianizationHomomorphism}}
\label{AbelianizationHomomorphism}
}\hfill{\scriptsize (function)}}\\
\textbf{\indent Returns:\ }
 the homomorphism from \mbox{\texttt{\mdseries\slshape F}} to $F/[F,F]$. 



 The argument of this method is a permutation group \mbox{\texttt{\mdseries\slshape F}} $\le S_{d}$. This method can be used as an example for the argument \mbox{\texttt{\mdseries\slshape rho}} in the methods \texttt{SpheresProduct} (\ref{SpheresProduct}) and \texttt{PI} (\ref{PI}). 

 }

 

 
\begin{Verbatim}[commandchars=!@|,fontsize=\small,frame=single,label=Example]
  !gapprompt@gap>| !gapinput@F:=PrimitiveGroup(5,3);|
  AGL(1, 5)
  !gapprompt@gap>| !gapinput@ab:=AbelianizationHomomorphism(PrimitiveGroup(5,3));|
  [ (2,3,4,5), (1,2,3,5,4) ] -> [ f1, <identity> of ... ]
  !gapprompt@gap>| !gapinput@Elements(Range(ab));|
  [ <identity> of ..., f1, f2, f1*f2 ]
  !gapprompt@gap>| !gapinput@StructureDescription(Range(ab));|
  "C4"
\end{Verbatim}
 

\subsection{\textcolor{Chapter }{SpheresProduct}}
\logpage{[ 4, 4, 3 ]}\nobreak
\hyperdef{L}{X83A7A23D875BFAA2}{}
{\noindent\textcolor{FuncColor}{$\triangleright$\enspace\texttt{SpheresProduct({\mdseries\slshape d, k, aut, rho, R})\index{SpheresProduct@\texttt{SpheresProduct}}
\label{SpheresProduct}
}\hfill{\scriptsize (function)}}\\
\textbf{\indent Returns:\ }
 the product $\prod_{r\in R}\prod_{x\in S(b,r)}$\mbox{\texttt{\mdseries\slshape rho}}$(\sigma_{1}($\mbox{\texttt{\mdseries\slshape aut}}$,x))\in\mathrm{im}($\mbox{\texttt{\mdseries\slshape rho}}$)$. 



 The arguments of this method are a degree \mbox{\texttt{\mdseries\slshape d}} $\in\mathbb{N}_{\ge 3}$, a radius \mbox{\texttt{\mdseries\slshape k}} $\in\mathbb{N}$, an automorphism \mbox{\texttt{\mdseries\slshape aut}} of $B_{d,k}$ all of whose $1$-local actions are in the domain of the homomorphism \mbox{\texttt{\mdseries\slshape rho}} from a subgroup of $S_d$ to an abelian group, and a sublist \mbox{\texttt{\mdseries\slshape R}} of \texttt{[0..k-1]}. This method is used in the implementation of \texttt{PI} (\ref{PI}). 

 }

 

 
\begin{Verbatim}[commandchars=!@|,fontsize=\small,frame=single,label=Example]
  !gapprompt@gap>| !gapinput@rho:=SignHomomorphism(SymmetricGroup(3));;|
  !gapprompt@gap>| !gapinput@SpheresProduct(3,2,gamma(2,3,(1,2)),rho,[0]);|
  (1,2)
  !gapprompt@gap>| !gapinput@SpheresProduct(3,2,gamma(2,3,(1,2)),rho,[0,1]);|
  ()
\end{Verbatim}
 

 
\begin{Verbatim}[commandchars=!@|,fontsize=\small,frame=single,label=Example]
  !gapprompt@gap>| !gapinput@F:=PrimitiveGroup(5,3);|
  AGL(1, 5)
  !gapprompt@gap>| !gapinput@rho:=AbelianizationHomomorphism(F);;|
  !gapprompt@gap>| !gapinput@Elements(Range(rho));|
  [ <identity> of ..., f1, f2, f1*f2 ]
  !gapprompt@gap>| !gapinput@StructureDescription(Range(rho));|
  "C4"
  !gapprompt@gap>| !gapinput@aut:=Random(F);|
  (1,2,4,5)
  !gapprompt@gap>| !gapinput@SpheresProduct(5,3,gamma(3,5,aut),rho,[2]);|
  <identity> of ...
  !gapprompt@gap>| !gapinput@SpheresProduct(5,3,gamma(3,5,aut),rho,[1,2]);|
  f1*f2
  !gapprompt@gap>| !gapinput@SpheresProduct(5,3,gamma(3,5,aut),rho,[0,1,2]);|
  f2
\end{Verbatim}
 

\subsection{\textcolor{Chapter }{PI}}
\logpage{[ 4, 4, 4 ]}\nobreak
\hyperdef{L}{X87EBFB50781D4B4D}{}
{\noindent\textcolor{FuncColor}{$\triangleright$\enspace\texttt{PI({\mdseries\slshape l, d, F, rho, R})\index{PI@\texttt{PI}}
\label{PI}
}\hfill{\scriptsize (function)}}\\
\textbf{\indent Returns:\ }
 the group $\Pi^{l}($\mbox{\texttt{\mdseries\slshape F}}$,$\mbox{\texttt{\mdseries\slshape rho}}$,$\mbox{\texttt{\mdseries\slshape R}}$)=\{\alpha\in\Phi^{l}(F)\mid \prod_{r\in R}\prod_{x\in S(b,r)}$\mbox{\texttt{\mdseries\slshape rho}}$(\sigma_{1}(\alpha,x))=1\}\le\mathrm{Aut}(B_{d,l})$. 



 The arguments of this method are a degree \mbox{\texttt{\mdseries\slshape l}} $\in\mathbb{N}_{\ge 2}$, a radius \mbox{\texttt{\mdseries\slshape d}} $\in\mathbb{N}_{\ge 3}$, a permutation group \mbox{\texttt{\mdseries\slshape F}} $\le S_d$, a homomorphism $\rho$ from \mbox{\texttt{\mdseries\slshape F}} to an abelian group that is surjective on every point stabilizer in \mbox{\texttt{\mdseries\slshape F}}, and a non-empty, non-zero subset \mbox{\texttt{\mdseries\slshape R}} of \texttt{[0..l-1]} that contains $l-1$. 

 }

 

 
\begin{Verbatim}[commandchars=!@|,fontsize=\small,frame=single,label=Example]
  !gapprompt@gap>| !gapinput@F:=PrimitiveGroup(5,3);|
  AGL(1, 5)
  !gapprompt@gap>| !gapinput@rho1:=AbelianizationHomomorphism(F);;|
  !gapprompt@gap>| !gapinput@rho2:=SignHomomorphism(F);;|
  !gapprompt@gap>| !gapinput@PI(3,5,F,rho1,[0,1,2]);|
  <permutation group with 4 generators>
  !gapprompt@gap>| !gapinput@Index(PHI(3,5,1,F),last);|
  4
  !gapprompt@gap>| !gapinput@PI(3,5,F,rho2,[0,1,2]);|
  <permutation group with 6 generators>
  !gapprompt@gap>| !gapinput@Index(PHI(3,5,1,F),last);|
  2
\end{Verbatim}
 }

 
\section{\textcolor{Chapter }{Semidirect products}}\label{Chapter_Examples_Section_Semidirect_products}
\logpage{[ 4, 5, 0 ]}
\hyperdef{L}{X87FE512E7DB7346C}{}
{
  When a subgroup $F\le\mathrm{Aut}(B_{d,k})$ satisfies (C) and admits an involutive compatibility cocycle $z$ (which is automatic when $k=1$) one can characterise the kernels $K\le\Phi_{k}(F)\cap\ker(\pi_{k})$ that fit into a $z$-split exact sequence $1\to K\to\Sigma(F,K)\to F\to 1$ for some subgroup $\Sigma(F,K)\le\mathrm{Aut}(B_{d,k+1})$ that satisfies (C). This characterisation is implemented in this section. 

 
\subsection{\textcolor{Chapter }{CompatibleKernels}}\label{CompatibleKernels}
\logpage{[ 4, 5, 1 ]}
\hyperdef{L}{X7F425DFC8760388F}{}
{
\noindent\textcolor{FuncColor}{$\triangleright$\enspace\texttt{CompatibleKernels({\mdseries\slshape d, F})\index{CompatibleKernels@\texttt{CompatibleKernels}!for d, F}
\label{CompatibleKernels:for d, F}
}\hfill{\scriptsize (operation)}}\\
\noindent\textcolor{FuncColor}{$\triangleright$\enspace\texttt{CompatibleKernels({\mdseries\slshape F, z})\index{CompatibleKernels@\texttt{CompatibleKernels}!for F, z}
\label{CompatibleKernels:for F, z}
}\hfill{\scriptsize (operation)}}\\


 

 
\begin{description}
\item[{for the arguments \mbox{\texttt{\mdseries\slshape d}}, \mbox{\texttt{\mdseries\slshape F}}}]  Returns: the list of kernels $K\le\prod_{\omega\in\Omega}F_{\omega}\cong\ker\pi\le\mathrm{Aut}(B_{d,2})$ that are preserved by the action \mbox{\texttt{\mdseries\slshape F}} $\curvearrowright\prod_{\omega\in\Omega}F_{\omega}$, $a\cdot(a_{\omega})_{\omega}:=(aa_{a^{-1}\omega}a^{-1})_{\omega}$. 

 The arguments of this method are a degree \mbox{\texttt{\mdseries\slshape d}} $\in\mathbb{N}_{\ge 3}$, and a permutation group \mbox{\texttt{\mdseries\slshape F}} $\le S_{d}$. The kernels output by this method are compatible with \mbox{\texttt{\mdseries\slshape F}} with respect to the standard cocycle (see \texttt{InvolutiveCompatibilityCocycle} (\ref{InvolutiveCompatibilityCocycle})) and can be used in the method \texttt{SIGMA} (\ref{SIGMA}). 
\item[{for the arguments \mbox{\texttt{\mdseries\slshape d}}, \mbox{\texttt{\mdseries\slshape k}}, \mbox{\texttt{\mdseries\slshape F}}, \mbox{\texttt{\mdseries\slshape z}}}]  Returns: the list of kernels $K\le\Phi_{k}(F)\cap\ker(\pi_{k})\le\mathrm{Aut}(B_{d,k+1})$ that are normalized by $\Gamma_{z}($\mbox{\texttt{\mdseries\slshape F}}$)$ and such that for all $k\in K$ and $\omega\in\Omega$ there is $k_{\omega}\in K$ with $\mathrm{pr}_{\omega}k_{\omega}=z(\mathrm{pr}_{\omega}k,\omega)^{-1}$. 

 The arguments of this method are a local action \mbox{\texttt{\mdseries\slshape F}} $\le\mathrm{Aut}(B_{d,k})$ that satisfies (C) and an involutive compatibility cocycle \mbox{\texttt{\mdseries\slshape z}} of \mbox{\texttt{\mdseries\slshape F}} (see \texttt{InvolutiveCompatibilityCocycle} (\ref{InvolutiveCompatibilityCocycle})). It can be used in the method \texttt{SIGMA} (\ref{SIGMA}). 
\end{description}
 

 }

 

 
\begin{Verbatim}[commandchars=!@|,fontsize=\small,frame=single,label=Example]
  !gapprompt@gap>| !gapinput@CompatibleKernels(3,SymmetricGroup(3));|
  [ Group(()), Group([ (1,2)(3,4)(5,6) ]), Group([ (3,4)(5,6), (1,2)(5,6) ]), 
    Group([ (5,6), (3,4), (1,2) ]) ]
\end{Verbatim}
 

 
\begin{Verbatim}[commandchars=!@|,fontsize=\small,frame=single,label=Example]
  !gapprompt@gap>| !gapinput@P:=SymmetricGroup(3);;|
  !gapprompt@gap>| !gapinput@rho:=SignHomomorphism(P);;|
  !gapprompt@gap>| !gapinput@F:=PI(2,3,P,rho,[1]);;|
  !gapprompt@gap>| !gapinput@z:=InvolutiveCompatibilityCocycle(3,2,F);;|
  [ Group(()), Group([ (1,2)(3,4)(5,6)(7,8)(9,10)(11,12) ]), 
    Group([ (1,2)(3,4)(5,6)(7,8), (5,6)(7,8)(9,10)(11,12) ]), 
    Group([ (5,6)(7,8), (1,2)(3,4), (9,10)(11,12) ]) ]
\end{Verbatim}
 
\subsection{\textcolor{Chapter }{SIGMA}}\label{SIGMA}
\logpage{[ 4, 5, 2 ]}
\hyperdef{L}{X823707DF821E79A0}{}
{
\noindent\textcolor{FuncColor}{$\triangleright$\enspace\texttt{SIGMA({\mdseries\slshape d, F, K})\index{SIGMA@\texttt{SIGMA}!for d, F, K}
\label{SIGMA:for d, F, K}
}\hfill{\scriptsize (operation)}}\\
\noindent\textcolor{FuncColor}{$\triangleright$\enspace\texttt{SIGMA({\mdseries\slshape F, K, z})\index{SIGMA@\texttt{SIGMA}!for F, K, z}
\label{SIGMA:for F, K, z}
}\hfill{\scriptsize (operation)}}\\


 

 
\begin{description}
\item[{for the arguments \mbox{\texttt{\mdseries\slshape d}}, \mbox{\texttt{\mdseries\slshape F}}, \mbox{\texttt{\mdseries\slshape K}}}]  Returns: the semidirect product $\Sigma($\mbox{\texttt{\mdseries\slshape F}}$,$\mbox{\texttt{\mdseries\slshape K}}$)\le\mathrm{Aut}(B_{d,2})$. 

 The arguments of this method are a degree \mbox{\texttt{\mdseries\slshape d}} $\in\mathbb{N}_{\ge 3}$, a subgroup \mbox{\texttt{\mdseries\slshape F}} of $S_{d}$ and a compatible kernel \mbox{\texttt{\mdseries\slshape K}} for \mbox{\texttt{\mdseries\slshape F}} (see \texttt{CompatibleKernels} (\ref{CompatibleKernels})). 
\item[{for the arguments \mbox{\texttt{\mdseries\slshape d}}, \mbox{\texttt{\mdseries\slshape k}}, \mbox{\texttt{\mdseries\slshape F}}, \mbox{\texttt{\mdseries\slshape K}}, \mbox{\texttt{\mdseries\slshape z}}}]  Returns: the semidirect product $\Sigma_{z}($\mbox{\texttt{\mdseries\slshape F}}$,$\mbox{\texttt{\mdseries\slshape K}}$)\le\mathrm{Aut}(B_{d,k+1})$. 

 The arguments of this method are a local action \mbox{\texttt{\mdseries\slshape F}} of $\mathrm{Aut}(B_{d,k})$ that satisfies (C) and a kernel \mbox{\texttt{\mdseries\slshape K}} that is compatible for \mbox{\texttt{\mdseries\slshape F}} with respect to the involutive compatibility cocycle \mbox{\texttt{\mdseries\slshape z}} (see \texttt{InvolutiveCompatibilityCocycle} (\ref{InvolutiveCompatibilityCocycle}) and \texttt{CompatibleKernels} (\ref{CompatibleKernels})) of \mbox{\texttt{\mdseries\slshape F}}. 
\end{description}
 

 }

 

 
\begin{Verbatim}[commandchars=!@|,fontsize=\small,frame=single,label=Example]
  !gapprompt@gap>| !gapinput@S3:=SymmetricGroup(3);;|
  !gapprompt@gap>| !gapinput@kernels:=CompatibleKernels(3,S3);|
  [ Group(()), Group([ (1,2)(3,4)(5,6) ]), Group([ (3,4)(5,6), (1,2)(5,6) ]), 
    Group([ (5,6), (3,4), (1,2) ]) ]
  !gapprompt@gap>| !gapinput@for K in kernels do Print(Size(SIGMA(3,S3,K)),"\n"); od;|
  6
  12
  24
  48
\end{Verbatim}
 

 
\begin{Verbatim}[commandchars=!@|,fontsize=\small,frame=single,label=Example]
  !gapprompt@gap>| !gapinput@P:=SymmetricGroup(3);;|
  !gapprompt@gap>| !gapinput@rho:=SignHomomorphism(P);;|
  !gapprompt@gap>| !gapinput@F:=PI(2,3,P,rho,[1]);;|
  !gapprompt@gap>| !gapinput@z:=InvolutiveCompatibilityCocycle(3,2,F);;|
  !gapprompt@gap>| !gapinput@kernels:=CompatibleKernels(3,2,F,z);|
  [ Group(()), Group([ (1,2)(3,4)(5,6)(7,8)(9,10)(11,12) ]), 
    Group([ (1,2)(3,4)(5,6)(7,8), (5,6)(7,8)(9,10)(11,12) ]), 
    Group([ (5,6)(7,8), (1,2)(3,4), (9,10)(11,12) ]) ]
  !gapprompt@gap>| !gapinput@for K in kernels do Print(Size(SIGMA(3,2,F,K,z)),"\n"); od;|
  24
  48
  96
  192
\end{Verbatim}
 }

 }

   
\chapter{\textcolor{Chapter }{Discreteness}}\label{Chapter_Discreteness}
\logpage{[ 5, 0, 0 ]}
\hyperdef{L}{X7875F15E81CDBD6C}{}
{
  

 This chapter contains functions that are related to the discreteness property
(D) presented in Proposition 3.12 of \cite{Tor20}. 
\section{\textcolor{Chapter }{The discreteness condition (D)}}\label{Section_condition_D}
\logpage{[ 5, 1, 0 ]}
\hyperdef{L}{X7B8BCB2681070C9C}{}
{
  Said proposition shows that for a given $F\le \mathrm{Aut}(B_{d,k})$ the group $\mathrm{U}_{k}(F)$ is discrete if and only if the maximal compatible subgroup $C(F)$ of $F$ satisfies condition (D): 
\[\forall \omega \in \Omega: F_{T_{\omega}}=\{\mathrm{id}\},\]
 where $T_{\omega}$ is the $k-1$-neighbourhood of the the edge $(b,b_{\omega})$ inside $B_{d,k}$. In other words, $F$ satisfies (D) if and only if the compatibility set $C_{F}(\mathrm{id},\omega)=\{\mathrm{id}\}$. We distinguish between $F$ satisfying condition (D) and $\mathrm{U}_{k}(F)$ being discrete with the methods \texttt{SatisfiesD} (\ref{SatisfiesD}) and \texttt{IsDiscrete} (\ref{IsDiscrete}) below. }

 
\section{\textcolor{Chapter }{Discreteness}}\label{Chapter_Discreteness_Section_Discreteness}
\logpage{[ 5, 2, 0 ]}
\hyperdef{L}{X7875F15E81CDBD6C}{}
{
  

\subsection{\textcolor{Chapter }{SatisfiesD (for IsLocalAction)}}
\logpage{[ 5, 2, 1 ]}\nobreak
\hyperdef{L}{X87A11A3E7BDC0549}{}
{\noindent\textcolor{FuncColor}{$\triangleright$\enspace\texttt{SatisfiesD({\mdseries\slshape F})\index{SatisfiesD@\texttt{SatisfiesD}!for IsLocalAction}
\label{SatisfiesD:for IsLocalAction}
}\hfill{\scriptsize (property)}}\\
\textbf{\indent Returns:\ }
 \texttt{true} if \mbox{\texttt{\mdseries\slshape F}} satisfies the discreteness condition (D), and \texttt{false} otherwise. 



 The argument of this attribute is a local action \mbox{\texttt{\mdseries\slshape F}} $\le\mathrm{Aut}(B_{d,k})$ (\texttt{IsLocalAction} (\ref{IsLocalAction})). 

 }

 

 
\begin{Verbatim}[commandchars=!@|,fontsize=\small,frame=single,label=Example]
  !gapprompt@gap>| !gapinput@G:=GAMMA(3,SymmetricGroup(3));|
  Group([ (1,4,5)(2,3,6), (1,3)(2,4)(5,6) ])
  !gapprompt@gap>| !gapinput@SatisfiesD(3,2,G);|
  true
\end{Verbatim}
 

\subsection{\textcolor{Chapter }{IsDiscrete (for IsLocalAction)}}
\logpage{[ 5, 2, 2 ]}\nobreak
\hyperdef{L}{X7C2F10AC79AC7213}{}
{\noindent\textcolor{FuncColor}{$\triangleright$\enspace\texttt{IsDiscrete({\mdseries\slshape F})\index{IsDiscrete@\texttt{IsDiscrete}!for IsLocalAction}
\label{IsDiscrete:for IsLocalAction}
}\hfill{\scriptsize (property)}}\\
\textbf{\indent Returns:\ }
 \texttt{true} if $\mathrm{U}_{k}(F)$ is discrete, and \texttt{false} otherwise. 



 The argument of this attribute is a local action \mbox{\texttt{\mdseries\slshape F}} $\le\mathrm{Aut}(B_{d,k})$ (\texttt{IsLocalAction} (\ref{IsLocalAction})). 

 }

 

 
\begin{Verbatim}[commandchars=!@|,fontsize=\small,frame=single,label=Example]
  !gapprompt@gap>| !gapinput@G:=GAMMA(3,SymmetricGroup(3));|
  Group([ (1,4,5)(2,3,6), (1,3)(2,4)(5,6) ])
  !gapprompt@gap>| !gapinput@IsDiscrete(3,2,G);|
  true
\end{Verbatim}
 

 
\begin{Verbatim}[commandchars=!@|,fontsize=\small,frame=single,label=Example]
  !gapprompt@gap>| !gapinput@IsDiscrete(3,2,Group((1,2)));|
  true
  !gapprompt@gap>| !gapinput@SatisfiesD(3,2,Group((1,2)));|
  false
  !gapprompt@gap>| !gapinput@C:=MaximalCompatibleSubgroup(3,2,Group((1,2)));|
  Group(())
  !gapprompt@gap>| !gapinput@SatisfiesD(3,2,C);|
  true
\end{Verbatim}
 }

 
\section{\textcolor{Chapter }{Cocycles}}\label{Chapter_Discreteness_Section_Cocycles}
\logpage{[ 5, 3, 0 ]}
\hyperdef{L}{X85A9B66278AF63D9}{}
{
  Subgroups $F\le\mathrm{Aut}(B_{d,k})$ that satisfy both (C) and (D) admit an involutive compatibility cocycle, i.e.
a map $z:F\times\{1,\ldots,d\}\to F$ that satisfies certain properties, see \cite[Section 3.2.2]{Tor20}. When $F$ satisfies just (C), it may still admit an involutive compatibility cocycle. In
this case, F admits an extension $\Gamma_{z}(F)\le\mathrm{Aut}(B_{d,k})$ that satisfies both (C) and (D). Involutive compatibility cocycles can be
searched for using \texttt{InvolutiveCompatibilityCocycle} (\ref{InvolutiveCompatibilityCocycle}) and \texttt{AllInvolutiveCompatibilityCocycles} (\ref{AllInvolutiveCompatibilityCocycles}) below. 

\subsection{\textcolor{Chapter }{InvolutiveCompatibilityCocycle (for IsLocalAction)}}
\logpage{[ 5, 3, 1 ]}\nobreak
\hyperdef{L}{X80ADE0E379590053}{}
{\noindent\textcolor{FuncColor}{$\triangleright$\enspace\texttt{InvolutiveCompatibilityCocycle({\mdseries\slshape F})\index{InvolutiveCompatibilityCocycle@\texttt{InvolutiveCompatibilityCocycle}!for IsLocalAction}
\label{InvolutiveCompatibilityCocycle:for IsLocalAction}
}\hfill{\scriptsize (attribute)}}\\
\textbf{\indent Returns:\ }
an involutive compatibility cocycle of \mbox{\texttt{\mdseries\slshape F}}, which is a mapping \mbox{\texttt{\mdseries\slshape F}}$\times$\texttt{[1..d]}$\to$\mbox{\texttt{\mdseries\slshape F}} with certain properties, if it exists, and \texttt{fail} otherwise. When \mbox{\texttt{\mdseries\slshape k}} $=1$, the standard cocycle is returned. 



 The argument of this attribute is a local action \mbox{\texttt{\mdseries\slshape F}} $\le\mathrm{Aut}(B_{d,k})$ (\texttt{IsLocalAction} (\ref{IsLocalAction})), which is compatible (\texttt{IsCompatible} (\ref{IsCompatible})). 

 }

 

 
\begin{Verbatim}[commandchars=!@|,fontsize=\small,frame=single,label=Example]
  !gapprompt@gap>| !gapinput@z:=InvolutiveCompatibilityCocycle(3,1,AlternatingGroup(3));|
  MappingByFunction( Domain([ [ (), 1 ], [ (), 2 ], [ (), 3 ], 
    [ (1,3,2), 1 ], [ (1,3,2), 2 ], [ (1,3,2), 3 ], [ (1,2,3), 1 ], 
    [ (1,2,3), 2 ], [ (1,2,3), 3 ] 
   ]), Alt( [ 1 .. 3 ] ), function( s ) ... end )
  !gapprompt@gap>| !gapinput@a:=Random(AlternatingGroup(3));; dir:=Random([1..3]);;|
  !gapprompt@gap>| !gapinput@a; Image(z,[a,dir]);|
  (1,3,2)
  (1,3,2)
\end{Verbatim}
 

 
\begin{Verbatim}[commandchars=!@|,fontsize=\small,frame=single,label=Example]
  !gapprompt@gap>| !gapinput@G:=GAMMA(3,AlternatingGroup(3));|
  Group([ (1,4,5)(2,3,6) ])
  !gapprompt@gap>| !gapinput@InvolutiveCompatibilityCocycle(3,2,G);|
  MappingByFunction( Domain([ [ (), 1 ], [ (), 2 ], [ (), 3 ], 
    [ (1,5,4)(2,6,3), 1 ], [ (1,5,4)(2,6,3), 2 ], [ (1,5,4)(2,6,3), 3 ], 
    [ (1,4,5)(2,3,6), 1 ], [ (1,4,5)(2,3,6), 2 ], [ (1,4,5)(2,3,6), 3 ] 
   ]), Group([ (1,4,5)(2,3,6) ]), function( s ) ... end )
  !gapprompt@gap>| !gapinput@InvolutiveCompatibilityCocycle(3,2,AutB(3,2));|
  fail
\end{Verbatim}
 

\subsection{\textcolor{Chapter }{AllInvolutiveCompatibilityCocycles (for IsLocalAction)}}
\logpage{[ 5, 3, 2 ]}\nobreak
\hyperdef{L}{X83A26CBF87AB1FD9}{}
{\noindent\textcolor{FuncColor}{$\triangleright$\enspace\texttt{AllInvolutiveCompatibilityCocycles({\mdseries\slshape F})\index{AllInvolutiveCompatibilityCocycles@\texttt{AllInvolutiveCompatibilityCocycles}!for IsLocalAction}
\label{AllInvolutiveCompatibilityCocycles:for IsLocalAction}
}\hfill{\scriptsize (attribute)}}\\
\textbf{\indent Returns:\ }
the list of all involutive compatibility cocycles of $F$. 



 The argument of this attribute is a local action \mbox{\texttt{\mdseries\slshape F}} $\le\mathrm{Aut}(B_{d,k})$ (\texttt{IsLocalAction} (\ref{IsLocalAction})), which is compatible (\texttt{IsCompatible} (\ref{IsCompatible})). 

 }

 

 
\begin{Verbatim}[commandchars=!@|,fontsize=\small,frame=single,label=Example]
  !gapprompt@gap>| !gapinput@S3:=SymmetricGroup(3);;|
  !gapprompt@gap>| !gapinput@Size(AllInvolutiveCompatibilityCocycles(3,1,S3));|
  4
  !gapprompt@gap>| !gapinput@Size(AllInvolutiveCompatibilityCocycles(3,2,GAMMA(3,S3)));|
  1
\end{Verbatim}
 }

 }

 \def\bibname{References\logpage{[ "Bib", 0, 0 ]}
\hyperdef{L}{X7A6F98FD85F02BFE}{}
}

\bibliographystyle{alpha}
\bibliography{UGALY.bib}

\addcontentsline{toc}{chapter}{References}

\def\indexname{Index\logpage{[ "Ind", 0, 0 ]}
\hyperdef{L}{X83A0356F839C696F}{}
}

\cleardoublepage
\phantomsection
\addcontentsline{toc}{chapter}{Index}


\printindex

\immediate\write\pagenrlog{["Ind", 0, 0], \arabic{page},}
\immediate\write\pagenrlog{["Ind", 0, 0], \arabic{page},}
\newpage
\immediate\write\pagenrlog{["End"], \arabic{page}];}
\immediate\closeout\pagenrlog
\end{document}
